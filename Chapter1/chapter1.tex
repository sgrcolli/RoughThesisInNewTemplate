%*******************************************************************************
%*********************************** First Chapter *****************************
%*******************************************************************************

\chapter{Introduction: The Aim Of VIDARR} \label{Chap:theAimOfVidarr} %Title of the First Chapter
The aim of the VIDARR project is to demonstrate the potential usefulness of a plastic scintillating detector for safeguarding purposes. To that end the characterisation and simulation of two versions of the detector will be used in order to show the full potential of this idea. Reactor monitoring using $\Bar{\nu_e}$ was suggested as early as 1978 \cite{Borovoi_1978} but the political climate of the cold war era prevent interest in the technology from being fully realised. In the modern political climate where nuclear power is seen as a stable form of low carbon power generation more nations are considering the technology once again. And as such the concern of the proliferation of atomic weapons has increased. Current methods of non-proliferation are dependent on accurate bookkeeping and empirical measurement estimations from power generation. Whilst these methods are effective more direct methods of measuring the flux from reactors and thereby the production of weapons grade material would greatly aid in the trust between nations and help to prevent the spread of atomic weaponry. 
\\\\This technology also has benefits for increasing the burn-up of nuclear waste which could increase power generation at nuclear power plants and reduce the production of nuclear waste which has to be stored. This potential is not yet fully realised due to the increased complexity of the method as it requires differentiating between different isotopes rather than just measuring flux. Tough this should be feasible it is not possible to test this capability until the upgraded VIDARR detector is deployed at a reactor site. 

\ifpdf
    \graphicspath{{Chapter1/Figs/Raster/}{Chapter1/Figs/PDF/}{Chapter1/Figs/}}
\else
    \graphicspath{{Chapter1/Figs/Vector/}{Chapter1/Figs/}}
\fi




