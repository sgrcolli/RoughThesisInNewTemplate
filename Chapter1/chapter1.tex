%*******************************************************************************
%*********************************** First Chapter *****************************
%*******************************************************************************

\ifpdf
    \graphicspath{{Chapter1/Figs/Raster/}{Chapter1/Figs/PDF/}{Chapter1/Figs/}}
\else
    \graphicspath{{Chapter1/Figs/Vector/}{Chapter1/Figs/}}
\fi

\chapter{Introduction to VIDARR and RMon} \label{Chap:theAimOfVidarr} %Title of the First Chapter

\section{Basic Overview of RMon and VIDARR}
The aim of the Verification Instrument for the Direct Assay of Radiation at Range (VIDARR) project is to demonstrate the potential usefulness of an anti-neutrino detector for safeguarding purposes. Between 2014 -- 2016 a prototype detector (RMon) based on the T2K ND 280 ECal \cite{Allan_2013} was deployed at the Wylfa reactor site. This deployment proved successful in measuring the reactor on period in 2014. However, there are improvements to the target mass, time resolution and cooling which would greatly improve these results. An upgrade that implements all of these improvements is planned and has been characterised through GEANT4 \cite{Agostinelli:2002hh} simulations. Reactor monitoring using $\Bar{\nu_e}$ was suggested as early as 1978 \cite{Borovoi_1978} but the political climate of the Cold War era prevented the approach from being pursued. In addition small silicon photon multipliers (SiPms) were not available which necessitated the use of large liquid tanks or large segments. In the modern political climate where nuclear power is seen as a stable form of low carbon power generation, more nations are considering the nuclear power once again. As such the concern of atomic weapons proliferation has increased. Current safeguards for non-proliferation are dependent on accurate bookkeeping and estimations from power generation. Whilst these methods are effective, more direct methods of measuring the flux from reactors and thereby the production of weapons-grade material would greatly aid international institutions and watchdogs. As well as preventing and helping to prevent the spread of atomic weaponry. 
\\\\Anti-neutrino reactor monitoring also has potential benefits for utilising fuel more effectively. Increasing power generation whilst decreasing the amount of nuclear waste to be stored. This potential is not yet fully realised due to the increased complexity of this method as it requires differentiating between different isotopes rather than just measuring reactor $\bar{\nu_e}$ flux. Whether this is feasible will be tested once the upgrades to VIDARR are completed and it is deployed at a new reactor site.  


\section{Anti-neutrino Reactor Monitoring}
% Obviously the two papers to mention here are vogal and beacom 1999 \cite{Vogel_1999} and muller et al 2011 \cite{Mueller_2011}. They cover the ground pretty well. \cite{Mueller_2011} has the $10^{20}$ $\Bar{\nu_e}$/s per GW$^{Th}$ and \cite{Vogel_1999} has the cross section values. \cite{Vogel_1999} is mostly dealing with stuff to first order, it may be worth addressing how this changes with increasing order. But probably not considering the low number of events anti-neutrino detection yields. Worth also going over the cowan and riens \cite{Cowan1956Confirmation} \cite{cowan1957test} again just briefly. The songs s1 project is also a necessity in 2007 it marked the beginning of serious reactor monitoring \cite{Bowden_2007}. And of course the original soviet paper which suggested this is as a possibility \cite{Borovoi_1978}. Also need to mention the IAEA and there workshop which suggested the limitations for this \cite{IAEA_2008}. 
% \\\\Probably best to go onto semi chronologically with the Cowan and Reins approach stating that reactor monitoring was one of the original approaches as fuel decays through beta decay (equation \ref{modern_beta_decay}) and the detection occurs through equation \ref{inverse_beta_decay}. It may be worth also mentioning section \ref{Direct_Measurements_section}. Then mentioning the soviet paper \cite{Borovoi_1978} then the move on to the first prototype with SONGS1 and then \cite{Bowden_2007} which then informed the IAEA report \cite{IAEA_2008}. Then talk about vogel \cite{Vogel_1999} and muller \cite{Mueller_2011}.
As mentioned earlier in section \ref{Direct_Measurements_section} an attempt to detect anti-neutrinos from reactors was the first method to prove anti-neutrinos existed. By measuring the response of anti-neutrinos in cadmium-doped liquid scintillator \cite{Cowan1956Confirmation} which used inverse $\beta$ decay (equation \ref{inverse_beta_decay}). $\sim$ 20 years after that experiment in 1978 it was proposed that measuring anti-neutrinos from reactors could be used for reactor monitoring \cite{Borovoi_1978}. However, due to the geopolitics of the time, the interest in non-proliferation was not as high as in modern times. It took until 2007 for the SONGS1 prototype to be deployed as proof of concept for reactor monitoring \cite{Bowden_2007}. 
\\\\After the deployment of the SONGS1 prototype the International Atomic Energy Agency (IAEA) issued a list of positive traits that they would like to see in an anti-neutrino reactor monitoring detector. These traits include inert construction, non-liquid, easy operation, cheap, portable, robust, above-ground operation, easy deployment \cite{IAEA_2008}. Whilst these are not requirements, all of these traits can be met in a single detector. For example, VIDARR and RMon meet all of these traits. As it they were purpose built to do so. 
\\\\Reactors produce O $\sim$ $10^{20}$ $\Bar{\nu_e}$/s per GW$^{Th}$ as the fission products in the fuel elements $\beta$ decay \cite{Mueller_2011}. Though these rates are very high, the interaction cross-section to first order is $\sim$ $10^{-42}$cm$^2$ as can be seen in figure \ref{vogelAndBeacomCrossSection} \cite{Vogel_1999}. In addition to this different fission, fractions will decay at different rates as can be seen in figure \ref{mullerAndVogelCombined}. The combination of these two effects can be seen in figure \ref{mullerEtAlDetectedSpectrum} \cite{Mueller_2011}. From figure \ref{mullerEtAlDetectedSpectrum} the range of anti-neutrino energies expected to be detected ranges from $\sim$ 1.8\,MeV to $\sim$ 8.5\,MeV. The threshold is 1.804\,MeV as this is the mass difference between the initial and final states \cite{Mueller_2011}.

\begin{figure}[!h]
\centering
\begin{minipage}{.45\textwidth}
  \centering
  \includegraphics[width=\linewidth]{Chapter1/Figs/Raster/vogelAndBeacomCrossSection.png}
  \captionof{figure}{Upper panel: total cross-section for Inverse $\beta$ decay; bottom panel: $\bra{}\cos(\theta)\ket{}$ for the same reaction; both as a function of the anti-neutrino energy. The solid line is the O(1/M) result and the short-dashed line is the O(1) result. The long-dashed line is the result of Eq. 3.18. From \cite{smith_1972} \cite{Vogel_1999}} 
  \label{vogelAndBeacomCrossSection}
\end{minipage}%
\qquad
\begin{minipage}{.45\textwidth}
  \centering
  \includegraphics[width=\linewidth]{Chapter1/Figs/Raster/mullerAndVogelCombined.png} 
  \captionof{figure}{A combination of the fission fractions from \cite{Mueller_2011} and the first-order approximation of the cross-section from \cite{Vogel_1999} and their evollution over the energy range where the two distributions cross over.}
  \label{mullerAndVogelCombined}
\end{minipage}
\end{figure}

\begin{figure}[!h]
 \centering
 \includegraphics[width=0.4\linewidth]{Chapter1/Figs/Raster/mullerEtAlDetectedSpectrum.png} %height of this plot had to be adjusted because of its unusal diemensions
 \captionof{figure}{Detected anti-neutrino spectrum for $^{235}$U fission (blue solid curve). Units are arbitrary and oscillation effects are suppressed. The detected rate rises from the threshold value at about 1.8\,MeV, reaches a maximum around 4\,MeV, and vanishes after 8\,MeV. This shape is the result of folding the emitted spectrum (black dashed-dotted curve), parametrization taken from \cite{Mueller_2011} inverse $\beta$ cross-section (red dashed curve). From \cite{Mueller_2011}} %~can be used as a kind of place holder in latex
 \label{mullerEtAlDetectedSpectrum}
\end{figure}

It may be possible to discern the different fission fraction isotopes from their detected spectra. Providing the energy resolution, statistics, and location of the detector is sufficient. This would be very useful for non-proliferation purposes as measuring the $^{239}$Pu burn-up would lead to a more accurate measure of weapons-grade material than measuring on/off cycles. As online refuelling methods seen in reactors such as the Advanced Gas-cooled Reactors (AGRs) would not work around the measurement of $^{239}$Pu burn-up. The AGR reactor type can refuel online without shutting down. Therefore measure the reactor on/off cycle may not be sufficient for that specific reactor type. This is also true for other types such as pebble beds. As such the measurement of $^{239}$Pu burn-up is desirable. 

\section{Existing Reactor monitoring programs}\label{sec:exisitingReactorMonitoringPrograms}
%I feel like Double Chooz, Reno, KamLAND and Daya Bay are all very similar, liquid scintillating detectors with Gd doping initlally trying to find theta_13 but then transitioned to the 5 MeV reactor bump all of these are not proliforation focussed but measrument focussed experiments 
%Double Chooz is probably one of the most well known reactor monitoring programs. \cite{abe2014improved} goes over some of the details and I have already mentioned it here. Also the \cite{lasserre2006} reference may be useful, used it once before in the first year literature survey. I've also cited \cite{Abe_2012} in the past and that may prove useful here. Further the reference \cite{Olive_2014} may also be of use. Double Chooz is a good one to cite because its clearly not in competition with VIDARR. It requires to be very close and a big hole dug underground and uses gadolinium suspended in liquid organic scintillator. Which is not cheap or portable, it does however give significantly better measurements for the disappearance of anti-neutrinos which the above references mention.
The Double Chooz experiment is an evolution of the original Chooz experiment which was set up at the Chooz nuclear power plant in France\cite{lasserre2006}. Both of these experiments attempted to measure the $\Bar{\nu_e}$ disappearance from the same reactor however the original Chooz experiment was unable to see any disappearance to a 90$\,\%$ confidence level \cite{Apollonio_2003}. Both experiments used gadolinium doped liquid scintillator with the Double Chooz experiment having a near and far detector at 280\,m and 1050\,m respectively\cite{lasserre2006}. Finally, in 2012 $\Bar{\nu_e}$ disappearance was observed at the Double Chooz experiment \cite{Abe_2012}. These results were improved further in 2014 \cite{abe2014improved}.
\begin{figure}[!h]
 \centering
 \includegraphics[width=0.25\linewidth]{Chapter2/Figs/Raster/DCNearDetector.png} %height of this plot had to be adjusted because of its unusal diemensions
 \captionof{figure}{The near detector of the Double Chooz experiment 250\,m - 300\,m from the reactor with an overburden of 30\,m of rock. From \cite{lasserre2006}.} %~can be used as a kind of place holder in latex
 \label{DoubleChoozNearDetector}
\end{figure}
\\\\The Double Chooz experiment uses detectors placed underground to allow for a large amount of overburden to reduce background rates. As seen in figure \ref{DoubleChoozNearDetector} the overburden for the near detector is 30\,m resulting in 300 water meter equivalent (w.m.e) \cite{lasserre2006}. This high overhburden was chosen to keep a high true-neutrino-signal to background ratio. The far detector only has 70-80\,m.w.e and so the outer shielding is different \cite{lasserre2006}. The Double Chooz measurement  was not the first to measure the rate of $\Bar{\nu_e}$ disappearance \cite{reno_may_2012} it was able to produce a spectrum of the $\Bar{\nu_e}$ disappearance for $\theta_{13}$ which can be seen in \ref{doubleChoozSpectrumNoCaption}. The measured energy spectrum which can be seen as black points in figure \ref{doubleChoozSpectrumNoCaption}, the closest match to the data in the double Chooz data was found to the fitted red line with a $\sin^2{2\theta_{13}}$ = 0.109 $\pm$0.030(stat)$\pm$0.025(syst) as seen in figure \ref{doubleChoozSpectrumNoCaption} \cite{Abe_2012}. The data exclude the no-oscillation hypothesis at 99.8$\%$ CL (2.9$\sigma$)\cite{Abe_2012}. This data was further expanded upon in 2014 producing a result of $\sin^2{2\theta_{13}}$ = 0.090$^{+0.032}_{-0.029}$ using 467.90 live days of data to within $3.0\sigma$ \cite{abe2014improved}.
\begin{figure}[!h]
 \centering
 \includegraphics[width=\linewidth]{Chapter2/Figs/Raster/doubleChoozSpectrumNoCaption.png} %Just use linewidth for this one
 \captionof{figure}{Measured prompt energy spectrum for each integration period (data points) superimposed on the expected prompt energy spectrum, including backgrounds (green region), for the no-oscillation (blue dotted curve) and best-fit (red solid curve) at $\sin^2{2\theta_{13}}$ = 0.109 and $\Delta$m$^2_{31}$ = 2.32$\times$10$^{-3}$eV$^2$. Inset: stacked spectra of backgrounds. Bottom: differences between data and no-oscillation prediction (data points), and differences between best fit prediction and no-oscillation prediction (red curve).The orange band represents the systematic uncertainties on the best-fit prediction. From \cite{Abe_2012}} %~can be used as a kind of place holder in latex
 \label{doubleChoozSpectrumNoCaption}
\end{figure}

%Papers found by this collaboration include \cite{reno_may_2012}, \cite{reno2013},  \cite{reno_may_2019}. 
RENO was the first collaboration to see $\Bar{\nu_e}$ disappearance \cite{Olive_2014}. The RENO experiment consisted of two detectors at a distance of 408.56\,m for the near detector, and 1443.99\,m for the far detector \cite{reno_may_2012}. A schematic of the detectors used in RENO can be seen in figure \ref{RENO_detector}, this detection system again uses Gd-doped liquid scintillator with PMTs. 

\begin{figure}[!h]
 \centering
 \includegraphics[width=0.5\linewidth]{Chapter2/Figs/Raster/RENO_detector.png} %Just use linewidth for this one

 \captionof{figure}{ schematic view of a RENO detector. The near and far detectors are identical. The detector seen in this figure consists of a main inner detector (ID) and an outer veto detector (OD) from \cite{reno_may_2012}} %~can be used as a kind of place holder in latex
 \label{RENO_detector}
\end{figure}

RENO's results in 2012 were consistent with neutrino oscillations to within 4.9 $\sigma$ from 6 2.8\,$GW_{th}$ reactors. The result from RENO in 2012 using a rate based analysis was $\sin^2{2\theta_{13}}$ = 0.113$\pm0.013$(stat.)$\pm0.019$(syst.) which was consistent with the findings that Double Chooz would produce later. These results were revised once again in 2013 using 800 days of live time to $\sin^2{2\theta_{13}}$ = 0.100 $\pm$ 0.010(stat) $\pm$ 0.015 (sys.) corresponding to 6.3 $\sigma$ significance \cite{reno2013}. The focus of the RENO collaboration has since shifted to analysing the 5\,MeV reactor anomaly \cite{reno_may_2019}.  
\\\\There are similar experiments to both Double Chooz and RENO. For example, the Daya Bay experiment in China uses similar technology to measure anti-neutrinos from reactors \cite{DayaBay2007Precision}. Similar to other experiments in this field the Daya Bay experiment has started to investigate the reactor anomaly at 5\,MeV \cite{Daya_Bay_2017}. Nucifer is also similar to the other experiments being a scaled-down version of Double Chooz, Daya Bay and RENO using Gd doped liquid scintillator with the goal of near field reactor monitoring \cite{nucifer2016}. Another experiment of interest is the WATCHMAN collaboration which has the goal of far-field reactor monitoring using $\bar{\nu_e}$ originally planned to be deployed in the U.S.A \cite{askins2015physics} it has since become a joint U.S, U.K project and will be deployed at Boulby mine \cite{burns2018remote}. Its technology is in a state of flux but will use Gd in a fluid detector. There has been some external collaboration between the VIDARR experiment and the WATCHMAN experiment. In particular, the application of the DANCE calorimeter data to form a DICEBOX for Gadolinium which VIDARR has also taken advantage of as part of this thseis/autor's work. % There is another paper for WATCHMAN talking about directionallity but this is a bit of a can of worms... \cite{danielson2019directionally} 
\\\\Whilst the previously mentioned experiments are a useful context for the VIDARR detector their goals and technology differ. The following experiments of PANDA, SoLid, and to some extent CHANDLER are more comparable. This is because they are solid plastic scintillators aiming for deployments at commercial reactor sites (at least for PANDA and SoLid). As such it is useful to see the compromises that these experiments have made in comparison to VIDARR. The Plastic Anti-Neutrino Detector Array (PANDA) is a segmented detector that uses plastic and Gadolinium shown in figure \ref{subFig:pandaClose}, the gadolinium's interaction with neutrons is shown in equation \ref{equ:gadolinumNAbsorption}. The bars clustered together can be seen in figure \ref{subFig:pandaFar}. The bars seen in figure \ref{subFig:pandaFar} are plastic scintillator with dimensions of 10\,cm $\times$ 10\,cm $\times$ 100\,cm with two 10\,cm\ $\times$ 10\,cm $\times$ 10\,cm acrylic cubic light guides are glued to both ends of the plastic scintillator with optical cement capped with photon multiplier tubes at either end of the bars for data collection \cite{PANDA_2014}. The interior of a plastic scintillating bar can be seen in figure \ref{subFig:pandaClose} which also shows the prompt and delayed event of an $\overline{\nu_e}$ event. In addition calibrations and comparison to Geant 4 \cite{Agostinelli:2002hh} have been performed with a baseline of a $^{60}$Co with reasonable agreement between the calibration source and the simulated response \cite{PANDA_2012}. 

\begin{equation}
n + {^{155,157}Gd} \rightarrow {^{156,158} Gd} + \gamma (\sim 8\,MeV)
\label{equ:gadolinumNAbsorption}
\end{equation}

\begin{figure}[!h]
\centering
\begin{subfigure}{.5\textwidth}
  \centering
  \includegraphics[width=\linewidth]{Chapter2/Figs/Raster/Panda_far.png}
  \captionsetup{width=.9\linewidth}
  \caption{}
  \label{subFig:pandaFar}
\end{subfigure}%
\begin{subfigure}{.5\textwidth}
  \centering
\includegraphics[width=\linewidth]{Chapter2/Figs/Raster/Panda_close.png}
  \captionsetup{width=.9\linewidth}
  \caption{}
  \label{subFig:pandaClose}
\end{subfigure}
\caption{The exterior of PANDA is shown in (a). The bars are clustered together with photon multiplier tubes at the end to measure photons. The interior is shown in (b) which also shows a  $\bar{\nu_e}$ interaction with prompt and delayed components. From \cite{PANDA_2014}.}
\label{fig:pandaCloseFar}
\end{figure}

The bars are not positioned perpendicularly in the $z$ direction (figure \ref{subFig:pandaFar}). Therefore tracking is more difficult in PANDA, due to not having specific $x$ and $y$ coordinates, this makes vetoing cosmic $\mu$ (a major source of background) potentially harder in PANDA than if the bars were perpendicular. The photon multiplier tubes (PMTs) used in PANDA are also expensive, the PANDA36 shown in figure \ref{subFig:pandaFar} is much smaller than the proposed final design of 100 channels and 1m$^3$ \cite{PANDA_2012} but PANDA is being built in stages because of the expense of these components. However, the potential for measuring many photons with high efficiency makes PMTs attractive. Both Lesser PANDA (16 channels) and PANDA36 have been deployed by van with background measurements and reactor measurements being taken from inside the van \cite{PANDA_2012}, \cite{PANDA_2014}. But due to the flammable nature of petrol and diesel, it is unclear whether reactor operators would be open to having a van stationed indefinitely outside their reactor buildings. PANADA36 has also been used for monitoring Terrestrial $\gamma$-ray Flashes (TGFs) from thunderstorms to allow for further background reduction at nuclear sites\cite{PANDA_tgf}. 
\\\\The SoLid detector is a plastic scintillating detector that uses wavelength shifting (WLS) fibres with MPPCs at the end to read out fibre signals and adjacent 5\,cm $\times$ 5\,cm $\times$ 5\,cm cubes being lit up at similar time intervals to discern particles \cite{Solid_proposal}. There is a prompt signal outputted from the positron in the plastic cubes and a delayed signal from the $^6$LiF:ZnS(Ag) screen interacting with neutrons as shown in figure \ref{fig:SolidCubeDiagram}. The prompt positron response and delayed neutron response is similar to the other experiments mentioned. In SoLid $^6$Li is used as the neutron capture agent (equation \ref{Li_interact_eq}), which produces an $\alpha$ and a 4.78\,MeV $\gamma$ \cite{Solid_readout}. The $\gamma$ signal is ignored because the $\alpha$ signal produced is clear and produces a specific pulse shape signal \cite{Solid_readout} which has a very minimal background due to its specific pulse shape. Setting it apart from other experiments that use the 8\,MeV Gd cascade. 

\begin{equation}
n + {^6Li} \rightarrow {^3H} + \alpha +4.78\,MeV
\label{Li_interact_eq}
\end{equation}

\begin{figure}[!h]
 \centering
 \includegraphics[height=58mm]{Chapter2/Figs/Raster/SoLid_cube.png}
 \captionof{figure}{SoLid cube during an inverse beta decay event anti-neutrino decays into a positron and neutron via equation \ref{inverse_beta_decay}, delayed neutron is captured by $^6$LiF:ZnS(Ag) screen, positron gives the prompt response. From \cite{Solid_readout}.} 
 \label{fig:SolidCubeDiagram}
\end{figure}

The challenge with using $^6$Li as a neutron capture agent is producing enough light for pulse shape discrimination (PSD) to be used in the analysis. The $\alpha$s need to be captured by the sulphur in the $^6$LiF:ZnS(Ag) screen which then emits a specific $\gamma$ into the plastic which in turn produces electrons via Compton scattering which are then captured by WLS fibres. The Wavelength shifting fibres may not capture enough photons for the pulse shape analysis to be used, an alternative method to this is that proposed by CHANDLER (Carbon Hydrogen Anti-Neutrino Detector with a Lithium Enhanced Raghavan-optical-lattice) \cite{aap2015}. Which uses PMT's and a Raghavan-optical-lattice to counter-act the low light issues in SoLid. However, the CHANDLER design also brings challenges as the detector cannot be too large otherwise attenuation and containment of events prevent signals from being read. 







