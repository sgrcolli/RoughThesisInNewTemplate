%!TEX root = ../thesis.tex
% ******************************* Thesis Appendix A ****************************
\chapter{Additional Machine Learning SVM Plots} 

\begin{figure}[!h]
\centering
\begin{minipage}{.45\textwidth}
  \centering
  \includegraphics[width=\linewidth]{Chapter4/Figs/adjustedSvmPlots/adjusted_LinSepExample.png}
  \captionof{figure}{How a support vector machine using LIBSVM trains on linear data with a separation.} 
  \label{fig:LinSepExample}
  \vspace{0.478cm}
\end{minipage}%
\qquad
\begin{minipage}{.45\textwidth}
  \centering
  \includegraphics[width=\linewidth]{Chapter4/Figs/adjustedSvmPlots/adjusted_LinNoSepExample.png}
  \captionof{figure}{How a support vector machine using LIBSVM trains on linear data with no separation. This is similar to how the neutron and noise data looks to the SVM in section \ref{sec:MachineLearningTrigger}.}
  \label{fig:LinNoSepExample}
\end{minipage}
\end{figure}

\begin{figure}[!h]
\centering
\begin{minipage}{.45\textwidth}
  \centering
  \includegraphics[width=\linewidth]{Chapter4/Figs/adjustedSvmPlots/adjusted_CircleSepExample.png}
  \captionof{figure}{How the a support vector machine using LIBSVM trains on circular data with a separation.} 
  \label{fig:CircleSepExample}
\end{minipage}%
\qquad
\begin{minipage}{.45\textwidth}
  \centering
  \includegraphics[width=\linewidth]{Chapter4/Figs/adjustedSvmPlots/adjusted_CircleNoSepExample.png}
  \captionof{figure}{How the a support vector machine using LIBSVM trains on circular data with no separation.}
  \label{fig:CircleNoSepExample}
\end{minipage}
\end{figure}

\begin{figure}[!h]
 \centering
 \includegraphics[width=0.5\linewidth]{Chapter4/Figs/adjustedSvmPlots/adjusted_exp_1GausseExample.png}
 \captionof{figure}{How the a support vector machine using LIBSVM trains on data with only a single Gaussian peak and exponential noise in the x and y.} 
 \label{fig:exp_1GausseExample}
\end{figure}

\begin{figure}[!h]
 \centering
 \includegraphics[width=\linewidth]{Appendix1/Figs/25OutOf425kSamples.png}
 \captionof{figure}{A Sample to show how the RBF kernel in figure \ref{fig:svmExp_GausseExamples} distorts the data to make it linearly separable. Every point in the training set (425 points) will be the maximum for a given slice and is distorted around. All of these slices are then analysed by the SVM at once. This figure represents the first 25 points that are distorted around. The SVM will view all of the 425 graphs simultaneously to form any complex boundary. But this causes the problem to jump from a n$^2$ computational problem (SVM alone) to an n$^3$ computational problem (SVM with kernel). } 
 \label{fig:25OutOf425kSamples}
\end{figure}

% \section*{Windows OS}

% \subsection*{TeXLive package - full version}
% \begin{enumerate}
% \item	Download the TeXLive ISO (2.2GB) from\\
% \href{https://www.tug.org/texlive/}{https://www.tug.org/texlive/}
% \item	Download WinCDEmu (if you don't have a virtual drive) from \\
% \href{http://wincdemu.sysprogs.org/download/}
% {http://wincdemu.sysprogs.org/download/}
% \item	To install Windows CD Emulator follow the instructions at\\
% \href{http://wincdemu.sysprogs.org/tutorials/install/}
% {http://wincdemu.sysprogs.org/tutorials/install/}
% \item	Right click the iso and mount it using the WinCDEmu as shown in \\
% \href{http://wincdemu.sysprogs.org/tutorials/mount/}{
% http://wincdemu.sysprogs.org/tutorials/mount/}
% \item	Open your virtual drive and run setup.pl
% \end{enumerate}

% \begin{table}
% \caption{A nice looking table}
% \centering
% \label{table:nice_table}
% \begin{tabular}{l c c c c}
% \hline 
% \multirow{2}{*}{Dental measurement} & \multicolumn{2}{c}{Species I} & \multicolumn{2}{c}{Species II} \\ 
% \cline{2-5}
%   & mean & SD  & mean & SD  \\ 
% \hline
% I1MD & 6.23 & 0.91 & 5.2  & 0.7  \\

% I1LL & 7.48 & 0.56 & 8.7  & 0.71 \\

% I2MD & 3.99 & 0.63 & 4.22 & 0.54 \\

% I2LL & 6.81 & 0.02 & 6.66 & 0.01 \\

% CMD & 13.47 & 0.09 & 10.55 & 0.05 \\

% CBL & 11.88 & 0.05 & 13.11 & 0.04\\ 
% \hline 
% \end{tabular}
% \end{table}


% \begin{table}
% \caption{Even better looking table using booktabs}
% \centering
% \label{table:good_table}
% \begin{tabular}{l c c c c}
% \toprule
% \multirow{2}{*}{Dental measurement} & \multicolumn{2}{c}{Species I} & \multicolumn{2}{c}{Species II} \\ 
% \cmidrule{2-5}
%   & mean & SD  & mean & SD  \\ 
% \midrule
% I1MD & 6.23 & 0.91 & 5.2  & 0.7  \\

% I1LL & 7.48 & 0.56 & 8.7  & 0.71 \\

% I2MD & 3.99 & 0.63 & 4.22 & 0.54 \\

% I2LL & 6.81 & 0.02 & 6.66 & 0.01 \\

% CMD & 13.47 & 0.09 & 10.55 & 0.05 \\

% CBL & 11.88 & 0.05 & 13.11 & 0.04\\ 
% \bottomrule
% \end{tabular}
% \end{table}


