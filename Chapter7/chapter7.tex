%*******************************************************************************
%******************************* Final Chapter *********************************
%*******************************************************************************

\chapter{Summary}

\ifpdf
    \graphicspath{{Chapter7/Figs/Raster/}{Chapter6/Figs/PDF/}{Chapter6/Figs/}}
\else
    \graphicspath{{Chapter7/Figs/Vector/}{Chapter7/Figs/}}
\fi

As more countries are considering nuclear power once again, the desire to cheaply and easily monitor reactors is also increasing. Atomic weapons proliferation is a major concern, as the creation of warheads is much easier once reactors have been constructed. Therefore, the goal to measure reactor operations non-invasively without reliance on book-keeping and reported power output is desirable. To this end the University Of Liverpool developed a prototype $\bar{\nu_e}$ detector to measure nuclear reactors in the near-field (O $\sim$ 10\,m -- 100\,m). This detector was designed to meet all of the positive traits the IAEA wanted to see in an $\bar{\nu_e}$ detector \cite{IAEA_2008}. This detector was officially nameless but was internally referred to as RMon and had 49 layers of plastic scintillating bars, with each bar measuring $152\,\textrm{cm} \times 4\,\textrm{cm} \times 1\,\textrm{cm}$.
\\\\RMon was based on the ND280 ECal, sharing the same sensors, scintillating bars, and WLS fibres. The electronics for RMon used the same boards as the ND280 ECal but the logic for the boards was significantly re-worked so that they could analyse IBD events. Further, lead sheets that had been in-between each alternating layer in the ND280 ECal to constrain noise particles were replaced with Gd sheets to absorb the neutrons from IBD events. The RMon detector was then stored inside a 20\,ft long ISO container that had basic cooling and a large fan to circulate air. Further, RMon possessed a water loop to cool the electronic boards and sensors. The detector itself was cladded in polythene shielding that was $\sim$ 10\,cm thick on the top and bottom faces and $\sim$ 5\,cm thick on the other faces. RMon's deployment at the Wylfa nuclear power plant from June 2014 to February 2016 proved successful, the reactor turn on was measured by Carroll et. al. \cite{Carroll_2018} and much cosmic $\mu$ data was stored and used for tomographic purposes later. This deployment proved that taking a heavily shielded pion beam detector and re-purposing it for above ground reactor operations was a viable approach for creating a $\bar{\nu_e}$ reactor monitoring detector. 
\\\\However, this successful deployment had room for improvement. The sensors were highly experimental first generation MPPCs and temperature stability was lacking in the containment. Due to these limitations spectral measurements from the $\bar{\nu_e}$ Wylfa data was not feasible. But as the PANDA collaboration has shown basic spectral measurement is possible with near-field $\bar{\nu_e}$ reactor monitoring \cite{IIRIE_Panda_2021}. In addition, many channels were uninstrumented and the shell of RMon could have housed more scintillator and cooling apparatus. For all of these reasons the upgrade from RMon to VIDARR was pursued to address these short comings. The VIDARR upgrade has progressed significantly, additional scintillator and cooling radiators have been added, new MPPCs have been connected and labelled, and a new shipping container with improved air flow has arrived in which the detector currently resides. This results in a detector which has $\sim$ 43\,\% more mass than RMon, significantly more water cooling, more stable temperature containment, whilst having one computer rack as opposed to two. Only RMon's shell, initial 49 layers of scintillator and corresponding WLS fibres, plastic connecting components and shielding have been kept. Everything else has been improved. The author of this thesis has been very involved with the upgrade process, helping to prepare and add the additional scintillator, attaching cables, and assembling components both PCB and plastic. 
\\\\However, most of the work done by the author has instead focused on Monte Carlo GEANT4 \cite{Agostinelli:2002hh} simulation, and data analysis. There are several reasons for this, in part it is due to the author being part of the LIV.DAT initiative and thus having a Ph.D. constructed around such tasks. Partially it is due to the outbreak of COVID-19. But mostly it is because the role of simulation and data analysis needed to be filled within the VIDARR collaboration, and it has been a pleasure to do so. As such, it would be prudent to surmise the work achieved by the author thus far now that context has been established. 
\\\\The simulation of the VIDARR detector has been greatly improved over the course of this Ph.D. Each component of the detector has been modelled more accurately. Thanks to the test stand data taken by George Holt the PE per MeV and attenuation per bar could be added into the simulation. In addition, the MPPC response and counting statistics have also been modelled and added in to the simulation. This allows for more realistic results from the simulation, which should correspond to the upgraded VIDARR detector. Not all improvements to the simulation were to increase accuracy. The decision to model the light output via Birks' law with the MINER$\nu$A Birks' constant was done to increase computational efficiency, as full light simulations are $\sim$ 1000 times more computational intensive. 
\\\\But perhaps the most important addition to the simulation is the DANCE Dicebox model for $^{157}$Gd. This model has the high energy $\gamma$-rays typical of the Gd cascade and conserves energy making it superior to the models found in GEANT4. The DANCE Dicebox always sums energy to $\sim$ 8\,MeV without breaking conservation of energy in the tail of the spectra < 3\,MeV. It would be highly advantageous in the future to also add in a DANCE style Dicebox for the $^{155}$Gd isotope. In absence of that the final state model was used for all other Gd isotopes. Whilst the final state model breaks the conservation of energy < 3\,MeV in the spectra it is a reasonable approximation as it attempts to match the final spectra produced by the Gd cascade. As the majority of neutrons will capture on $^{157}$Gd \cite{Abdushukurov_2010} this final state-Dicebox cascade hybrid model is still much improved compared to the base GEANT4 models. This is especially important as the Gd cascade serves as the trigger for the IBD events. In order to find the best possible separating boundary between the cascade trigger signal and conservative noise estimations a machine learning study using an SVM was performed. The SVM was able to find a linear boundary which had a good efficiency (75\,\%) and fantastic purity (92\,\%) (1 million signal events to 4 million noise events). This was a significant improvement moving from the by-hand selection criteria (cuts) whilst also reducing dimensionality from 3D to 2D. By utilising this machine learning technique it was possible to see how each of the dimensions inter-played with each other. Neutron trigger thresholds of 0.1\,MeV and 0.5\,MeV have proven very successful in simulation, others have been trailed but no improvement was found. 
\\\\Finally, due to the COVID-19 pandemic that started in November of 2019 the Ph.D. had to change direction as the VIDARR upgrade paused. This afforded the opportunity to analyse cosmic $\mu$ events that were included accidentally by the trigger at the Wylfa reactor site by RMon. Most of these events were unusable as many of the cycles were biased towards the trigger cycle but a selection of $\sim 3\times 10^6$ events were still usable. These events correspond to $\sim$ 3 hours of live time and were then used to perform cosmic $\mu$ tomography on the reactor site buildings. By utilising a control data set taken at Liverpool and taking the ratio of the two data sets shadows for the main building, turbine hall, reactor core / reactor core shielding, service towers, service buildings and steam bridges were seen. This was achieved by simulating each building in turn and adding to the composite shadow one building at a time in simulation. When comparing the outlines produced by the simulated composite shadow to the measured composite shadow a strong agreement can be seen. Although at the edges of each building scattering does cause a ``blurring'' effect. This strong agreement was further reinforced when the CRY \cite{ieee_cry_2007} $\theta$ distribution used in combination with GEANT4 confirmed that the Liverpool data set had no obvious shadows but Wylfa data set had blurred but clear deficits. This is unsurprising as the detector is highly segmented thus allowing for excellent reconstruction of the ($\phi,\theta$) space. This is especially true for the $\theta$ distribution as the segments are only 1\,cm thick in the z axis.
\\\\Cosmic $\mu$ tomography is a strong addition to VIDARR's capabilities as now the detector can image it's surroundings and unlike RMon, VIDARR will have a dedicated cosmic $\mu$ mode. This combined with the GEANT4 simulation will allow the VIDARR detector to infer its position relative to its surroundings to within a few m. This is especially true if the VIDARR detector is moved from its monitoring position without the consent of its operators. As $\bar{\nu_e}$ cannot be shielded against cosmic $\mu$ tomography further increases the safeguarding credentials of the VIDARR detector. This should also be possible with relatively small amounts of data that correspond to $\sim$ 1 hour of live time. These analysis techniques also benefit from multi-threading and a fast fitting time of < 2\,ms for > 99\,\% of cosmic $\mu$ candidates. Now the utility of cosmic $\mu$ events has been verified better control data sets will be taken in future. Once deployed the detector will activate cosmic $\mu$ mode for 1 hour a week to detect if it has been moved. 1 hour a week should be sufficient to guard against forced relocation without interrupting $\bar{\nu_e}$ measurements.
\\\\Finally, the VIDARR detector is about to switch on in the first half of 2022. With the upgrade nearing its end point a future deployment measuring atomic waste at Sellafield is planned. VIDARR's future post deployment is bright as its full capabilities will come to light, especially once it deploys at a reactor site. 
% The primary focus of RMon and VIDARR is the measurement of $\Bar{\nu_e}$s from reactors. Whilst RMon did measure $\Bar{\nu_e}$ data from 07-07-2014 to 25-02-2016 at the Wylfa reactor site (see figure \ref{fig:prototypeMeasumentFlux}) the results had a high noise level due to the condition of the electronics and the container being sub-optimal. Despite this RMon was able to determine the reactor on cycle between July 2014 an August 2014. Higher fidelity electronics and improved container cooling is planned (see section \ref{sec:DetectorConstruction}) and should greatly improve the signal to noise ratio. Thus VIDARR's results once deployed should be much improved when compared to RMon's. It is hoped that this improvement to the signal to noise ratio may even allow for the verification of different isotopes present in reactor fuel. In addition, there were only 1793 channels that were instrumented out of a possible 2660 channels, only $\sim$ 67\,\% of the detector. Not only would the number of $\Bar{\nu_e}$ candidates improve but also the quality of the containment of the trigger signal the gadolinium cascade (see figure \ref{fig:containment_comparison}). The upgrade seeks to address all of these issues increasing the overall signal to noise and increasing the number of signal events by $\sim$ 50\,\% from $\sim$ 200 (see figure \ref{fig:prototypeMeasumentFlux}) to $\sim$ 300. There are many experiments that are also looking at $\Bar{\nu_e}$s from reactors (see section \ref{sec:exisitingReactorMonitoringPrograms}). The number of these experiments suggest that this will be a healthy and productive field for years to come.
% \\\\In order to quantify the capabilities of the upgraded detector, the GEANT4 simulation has been continuously worked on and improved in tandem with the upgrade. Taking into account the detector effects such as the MPPC response, dark noise (see figure \ref{fig:fitting_of_non_peak_dark_noise}), counting statistics (see figure \ref{fig:CoutingStats10}), attenuation (see figure \ref{fig:attenuationPlot}), and quenching in the plastic (see section \ref{sec:GEANT4Simulation_quenchingLoss}). Unfortunately, the ongoing pandemic caused by COVID-19 has restricted the supply of electronics thus upgrading the electronics has taken far longer than expected. As a result, there has been much focus on improving the simulation, such as improving the Gd cascade (see section \ref{sec:GEANT4Simulation_modellingGadolinium}) and testing machine learning techniques on that generated data (see section \ref{sec:MachineLearningTrigger}). But the biggest change to this thesis has been the large focus on cosmic $\mu$ tomography as opposed to analysing $\Bar{\nu_e}$ data.
% \\\\Cosmic $\mu$ tomography at the Wylfa reactor site which is covered in chapter  \ref{chp:cosmicMuonTomography} and section \ref{sec:ReactorShadowMethodology} specifically shows how complicated the process can be. However, by utilising the GEANT4 simulation of the detector it is possible to get outlines of the buildings (figure \ref{fig:simulatedTrackerRecon}) and these outlines can then be overlaid on top of measured data (figure \ref{fig:measuredTrackerRecon}). Figures \ref{fig:simulatedTrackerRecon} and \ref{fig:measuredTrackerRecon} show how the buildings at the Wylfa reactor site project onto the RMon detector and how well this matches the measured shadows at Wylfa. This even allows the near reactor core to be resolved or at least it's containment/shielding. This was achieved by taking the ``transmission'' of how many cosmic $\mu$ pass through the buildings at the reactor site similar to the  MU-RAY collaboration's analysis of Mt. Vesuvius \cite{Ambrosino_2014}.    
% \\\\The analysis required a custom coded tracker built around the data from the RMon detector. Whilst using generated data to investigate and prevent potential biasing (see section \ref{sec:SimulationOfCosmics}). This tracker uses a simplex minimiser from the GNU scientific library \cite{galassi2002gnu} which forms the basis for most of the fitting and has no visible biasing associated with it (see figures \ref{fig:wylfaSideABHits}, \ref{fig:liverpoolSideABHits}). This tracker will be used both for cosmic $\mu$ tomography and calibration, including real-time online analysis which may require a time selection criteria to keep fittings below 3600 micro seconds. 
% \\\\Cosmic $\mu$ tomography should augment the VIDARR detector's capabilities as now it is possible for the detector to determine whether or not it has been moved. This is important as being moved will impact the measurement of $\Bar{\nu_e}$s and so will counter unscrupulous reactor sites that seek to move the detector in an attempt to give false readings. With the upgrade, the VIDARR detector should be able to give more accurate $\Bar{\nu_e}$ readings and $\mu$ readings due to the increase in mass and improvements in electronics and cooling. As well as justifying a cosmic $\mu$ mode leading to a higher cosmic $\mu$ rate as this work has shown the value of cosmic $\mu$ for tomographic purposes. 