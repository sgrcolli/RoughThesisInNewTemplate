%*******************************************************************************
%******************************* Final Chapter *********************************
%*******************************************************************************

\chapter{Summary}

\ifpdf
    \graphicspath{{Chapter7/Figs/Raster/}{Chapter6/Figs/PDF/}{Chapter6/Figs/}}
\else
    \graphicspath{{Chapter7/Figs/Vector/}{Chapter7/Figs/}}
\fi

The Primary focus of VIDARR is the measurement of $\Bar{\nu_e}$s from reactors. Whilst a prototype of the detector did measure $\Bar{\nu_e}$ data from 07-07-2014 to 25-02-2016 at the Wylfa reactor site (see figure \ref{fig:prototypeMeasumentFlux}) the results had a high noise level due to the condition of the electronics and the container being sub-optimal. Both of which will be improved with higher fidelity electronics and improved container cooling as described in section \ref{sec:DetectorConstruction}. In addition, there were only 1793 channels that were instrumented out of a possible 2660 channels, only $\sim$ 67\,\% of the detector. Not only would the number of $\Bar{\nu_e}$ candidates improve but also the quality of the containment of the trigger signal the gadolinium cascade (see figure \ref{fig:containment_comparison}). The upgrade seeks to address all of these issues increasing the overall signal to noise and increasing the number of signal events by $\sim$ 50\,\% from $\sim$ 200 (see figure \ref{fig:prototypeMeasumentFlux}) to $\sim$ 300. There are many experiments that are also looking at $\Bar{\nu_e}$s from reactors (see section \ref{sec:exisitingReactorMonitoringPrograms}). The number of these experiments suggest that this will be a healthy and productive field for years to come.
\\\\In order to quantify the capabilities of the upgraded detector, the GEANT4 simulation has been continuously worked on and improved in tandem with the upgrade. Taking into account the detector effects such as the MPPC response, dark noise (see figure \ref{fig:fitting_of_non_peak_dark_noise}), counting statistics (see figure \ref{fig:CoutingStats10}), attenuation (see figure \ref{fig:attenuationPlot}), and quenching in the plastic (see section \ref{sec:GEANT4Simulation_quenchingLoss}). Unfortunately, the ongoing pandemic caused by COVID-19 has restricted the supply of electronics thus fixing an electrical production mistake has taken far longer than expected. As a result, there has been much focus on improving the simulation, such as improving the Gd cascade (see section \ref{sec:GEANT4Simulation_modellingGadolinium}) and testing machine learning techniques on that generated data (see section \ref{sec:MachineLearningTrigger}). But the biggest change to this thesis has been the large focus on cosmic $\mu$ tomography as opposed to analysing $\Bar{\nu_e}$ data.
\\\\Cosmic $\mu$ tomography at the Wylfa reactor site which is covered in chapter  \ref{chp:cosmicMuonTomography} and section \ref{sec:ReactorShadowMethodology} specifically shows how complicated the process can be. However, by utilising the GEANT4 simulation of the detector it is possible to get outlines of the buildings (figure \ref{fig:simulatedTrackerRecon}) and these outlines can then be overlaid on top of measured data (figure \ref{fig:measuredTrackerRecon}). Figures \ref{fig:simulatedTrackerRecon} and \ref{fig:measuredTrackerRecon} show how the buildings at the Wylfa reactor site project onto the VIDARR prototype and how well this matches the measured shadows at Wylfa. This even allows the near reactor core to be resolved or at least it's containment/shielding. This was achieved by taking the ``transmission'' of how many cosmic $\mu$ pass through the buildings at the reactor site similar to the  MU-RAY collaboration's analysis of mt. Vesuvius \cite{Ambrosino_2014}.    
\\\\The analysis required a custom coded tracker built around the data from the prototype detector. Whilst using generated data to investigate and prevent potential biasing (see section \ref{sec:SimulationOfCosmics}). This tracker uses a simplex minimiser from the GNU scientific library \cite{galassi2002gnu} which forms the basis for most of the fitting and has no visible biasing associated with it (see figures \ref{fig:wylfaSideABHits}, \ref{fig:liverpoolSideABHits}). This tracker will be used both for cosmic $\mu$ tomography (which includes showers) and calibration (which excludes showers). 
\\\\Cosmic $\mu$ tomography should augment the VIDARR detector's capabilities as now it is possible for the detector to determine whether or not it has been moved. This is important as being moved will impact the measurement of $\Bar{\nu_e}$s and so will counter unscrupulous reactor sites that seek to move the detector in an attempt to give false readings. With the upgrade, the VIDARR detector should be able to give more accurate $\Bar{\nu_e}$ readings and $\mu$ readings due to the increase in mass and improvements in electronics and cooling. 