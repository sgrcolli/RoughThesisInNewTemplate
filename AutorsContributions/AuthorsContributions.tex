\chapter{Author's Contribution}

By it's very nature the scientific process is collaborative the work presented in this thesis is built on the work that has come before and during the work presented. 
\\\\The initial RMon deployment at Wylfa was done by Jon Coleman, Carl Metelko, Mathew Murdock and Yan-Jie Schnellbach from June 2014 to February 2016. The calibration of the data from wylfa was done by Mathew Murdock. The analysis of the calibrated data for cosmic $\mu$ tomography was done by the author (Ronald Collins). Whereas the analysis for the anti-neutrinos was done by Matt Murdock and Yan-Jie Schnellbach. 
\\\\The RMon detector is currently in the processes of being upgraded into the VIDARR detector. The electronics for the upgrade have been handled by Carl Metelko. The new contianment has been handled by Oliver Caunt. Overall management has been uder the watchfull eye of Jon Coleman. Whereas the assembly has mostly been handled by myself and Yan-Jie schnellbach. The assembly has included cutting scintillator, painting the ends of bars, putting scintillator in the detector housing, labelling cables, assembling PCBs, putting the electronics in the detector, 3D printing fibre holders and sheaths, assembling the multi pixel photon counters into their holders, and attaching and labelling the 2660 coaxial cables that connect the electronics to the detector's instruments. In addition Kieran Bridges has been behind the improvements to cooling including the new radiators which I have also helped with. 
\\\\The GEANT4 simulation for a single bar and the RMon detector was originally started by Matt Murdock and Yan-Jie Schellbach. I have extended both simulations by quantifying the response of the detector in simulation. As well as adding shielding, dead channels, TiO$_2$ coating around the bars. Further, I have expanded the cosmic $\mu$  simulation for tomographic purposes including a more realistic distribution from the CRY library and simulating the Wylfa reactor buildings. I added data driven effects such as dark noise and attenuation which used measurements from a single bar test stand that George Holt worked on. 
\\\\I have been the sole coder of the VIDARR analysis chain which has inspired by the RMon analysis chain created by Matt Murdock and Yan-Jie Schnellbach. This analysis chain can analyse measured data from RMon and simulated data from both RMon and VIDARR. The single bar simulations are analysed by separate code also written by me. 
\\\\A Brief Overview of the Thesis is as Follows:
\\\\\textbf{Introduction To Reactor Monitoring} -- A brief description of reactor monitoring principles and experiments involved in the field. Then a quick overview of VIDARR and RMon goals.  
\\\\\textbf{The Neutrino} -- A discussion of how the neutrino was discovered both through inference and direct measurements. As well an overview of neutrino oscillation.
\\\\\textbf{The RMon and VIDARR Detectors} -- An overview of RMon's technology and results. Then a look at the detector upgrade of RMon to VIDARR. And a basic look at background at reactor sites.
\\\\\textbf{Detector Component Simulation} -- Each component of the detector has been modelled either to improve performance as with characterising GEANT4's light model or to improve accuracy of the simulation such as modelling the attenuation, MPPC response, TiO2 coating, counting statistics and upgrading the Gd model. 
\\\\\textbf{Full Detector Simulation And Evaluation} -- The mass increase from RMon to VIDARR is quantified for both positrons and neutrons. The trigger response from neutrons is then analysed and subjected to a machine learning algorithm (an SVM) to find the best separation. Then cosmic $\mu$ are modelled to produce a tracker for cosmic $\mu$ tomography.
\\\\\textbf{cosmic $\mu$ Tomography} -- A brief overview of the cosmic $\mu$ tomography field including atmospheric production and particular fields that use tomographic techniques the RMon and VIDARR detectors could utilise. 
\\\\\textbf{Cosmic $\mu$ Tomography at Wylfa} -- RMon was deployed at Wylfa for $\sim$ 18 months. Cosmic $\mu$ were taken in accidental coincidence allow and for the resolution of specific buildings including the main reactor building, service buildings, steam bridges, turbine hall, and reactor core / reactor core shielding.