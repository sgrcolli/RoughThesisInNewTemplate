\chapter{Author's Contribution}

By it's very nature the scientific procsess is collaborative the work presentd in his thesis is built on the work that has come befor and during the work presented. 
\\\\The initial RMon deployment at Wylfa was done by Jon Coleman, Carl Metelko, Mathew Murdock and Yan-Jie Schnellbach from June 2016 to February 2018. The calibration of the data from wylfa was done by Mathew Murdock. The analysis of the calibrated data for cosmic $\mu$ tomography was done by me (Ronald Collins). Whereas the analysis for the $\Bar{\nu_e}$ was done by Matt Murdock and Yan-Jie Schnellbach. 
\\\\The RMon detector is currently in the processes of being upgraded into the VIDARR detector. The electronics for the upgrade have been handled by Carl Metelko. The new contianment has been handled by Oliver Caunt. Overal managment has been uder the watchfull eye of Jon Coleman. Whereas the assumbly has mostly been handled by myself and Yan-Jie schnellbach. The assembly has included cutting scintillator, painting the ends of bars, putting scintillator in the detector housing, labelling cables, assebmling PCBs, putting the electronics in the detector, 3D pringing fibre holders and sheaths, assembling the multi pixel photon counters into their holders, and attaching and labelling the 2660 coaxial cables that connect the electronics to the detector's instruments. In addition kieran bridges has been behind the improvements to cooling including the new radiators which I have also helped with. 
\\\\The simulation for a single bar and the whole detector was originally started by Matt Murdock and Yan-Jie Schellbach.  I have extended the simulation by quantifying the responce of the detector in simulation. Adding shielding, dead channels, TiO$_2$ coating around the bars. Further I have expanded the cosmic $\mu$  simulation for tomographic purposes including a more realistic distribution and simulating reactor buildings in simulation. The I added data driven effects such as dark noise and attenuation which used measurements from the single bar test stand that George Holt took. 
\\\\I have been the sole coder of the VIDARR analysis chain which has inspired by the RMon analysis chain created by Matt Murdock and Yan-Jie Schnellbach. This analysis chain can analyse measured dat from RMon and simulated data from both RMon and VIDARR. The single bar simulations are analysed by separate code also written by me. 
\\\\A Brief Overview of the Thesis is as Follows:
\\\\\textbf{Introduction To VIDARR And RMon} -- A quick overview of VIDARR and RMon's goals. Then a brief description of reactor monitoring principles and experiments involved in the field. 
\\\\\textbf{A Brief History Of Neutrinos} -- A discussion of how the neutrino was discovered both through inference and direct measurements. As well an overview of neutrino oscillation.
\\\\\textbf{The RMon and VIDARR Detectors} -- The basic structure of RMon and VIDARR including the upgrade made to RMon resulting in VIDARR. And a basic look at background at reactor sites.
\\\\\textbf{GEANT4 Simulation} -- A detialed simulation has been made including data driven effects for the VIDARR detector. Physical effects such as light output and Gd Cascade have been modelled as well. 
\\\\\textbf{Cosmic $\mu$ Telescopes} -- A quick look at the DIAPHANE and MU-RAY detectors and how their techniqes may or may not be suitable for RMon/VIDARR 
\\\\\textbf{Cosmic $\mu$ Tomography at Wylfa Using RMon} -- RMon was deployed at Wylfa for $\sim$ 18 months. Cosmic $\mu$ taken in accidental coincidence allow for the resolution of specific buildings including the main building, service buildings, turbine hall and reactor core / reactor core shielding.