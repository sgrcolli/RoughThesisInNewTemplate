\chapter{Author's Contribution}

By its very nature the scientific process is collaborative the work presented in this thesis is built on the work that has come before and during the work presented. 
\\\\The initial deployment at the Wylfa Magnox Nuclear Power Plant was done by Jon Coleman, Carl Metelko, Matthew Murdoch and Yan-Jie Schnellbach from June 2014 to February 2016. The calibration of the data from Wylfa was done by Matthew Murdoch. The analysis of the calibrated data for cosmic muon tomography was done by the author (Ronald Collins). Whereas the analysis for the anti-neutrino data was done by Matt Murdoch and Yan-Jie Schnellbach. 
\\\\The detector is currently in the processes of being upgraded into the VIDARR detector. The electronics for the upgrade have been handled by Carl Metelko. The new mobile laboratory container has been handled by Oliver Caunt. Overall management has been under the watchful eye of Jon Coleman. Whereas the assembly has mostly been handled by myself and Yan-Jie Schnellbach. The assembly has included cutting scintillator, painting the ends of bars, putting scintillator in the detector housing, labelling cables, assembling PCBs, putting the electronics in the detector, 3D printing fibre holders and sheaths, assembling the multi pixel photon counters into their holders, and attaching and labelling the 2660 coaxial cables that connect the electronics to the detector's instruments. In addition, Kieran Bridges has been behind the improvements to cooling including the new radiators which I have also helped with. 
\\\\The GEANT4 simulation for a single bar and the detector was originally started by Matt Murdoch and Yan-Jie Schnellbach. I have extended both simulations by quantifying the response of the detector in simulation. As well as adding shielding, dead channels, TiO$_2$ coating around the bars. Further, I have expanded the cosmic muon simulation for tomographic purposes including a more realistic distribution from the CRY library and simulating the Wylfa reactor buildings. I added data-driven effects such as dark noise and attenuation which used measurements from a single bar test stand that George Holt worked on. 
\\\\I have been the sole coder of the VIDARR analysis chain which has inspired by the initial analysis chain created by Matt Murdoch and Yan-Jie Schnellbach. This analysis chain can analyse measured data and simulated data from both iterations of the detector. The single bar simulations are analysed by separate code also written by me. 
\\\\A Brief Overview of the Thesis is as Follows:
\\\\\textbf{Introduction To Reactor Monitoring} -- A brief description of reactor monitoring principles and experiments involved in the field. Then a quick overview of VIDARR's goals.  
\\\\\textbf{Overview of Neutrino Physics} -- A discussion of how the neutrino was discovered both through inference and direct measurements. As well an overview of neutrino oscillation.
\\\\\textbf{The VIDARR Detector} -- An overview of the detector technology and results. Then a look at the detector upgrade to VIDARR. And a basic look at background at reactor sites.
\\\\\textbf{Detector Component Simulation} -- Each component of the detector has been modelled either to improve performance as with characterising GEANT4's light model or to improve accuracy of the simulation such as modelling the attenuation, MPPC response, TiO2 coating, counting statistics and upgrading the Gd model. 
\\\\\textbf{Full Detector Simulation And Evaluation} -- The mass increase upgrades to VIDARR is quantified for both positrons and neutrons. The trigger response from neutrons is then analysed and subjected to a machine learning algorithm (an SVM) to find the best separation. Then cosmic muon are modelled to produce a tracker for cosmic muon tomography.
\\\\\textbf{Cosmic Muon Tomography} -- A brief overview of the cosmic muon tomography field including atmospheric production and particular fields that use tomographic techniques the VIDARR detector could utilise. 
\\\\\textbf{Cosmic Muon Tomography at Wylfa} -- Initially the detector was deployed at the Wylfa reactor site for $\sim$ 18 months. Using the cosmic muon data collected over this time period it is possible to discern specific reactor buildings and features including the main reactor building, service buildings, steam bridges, turbine hall, and reactor core / reactor core shielding.