%*******************************************************************************
%******************************* Final Chapter *********************************
%*******************************************************************************

\chapter{Summary}

\ifpdf
    \graphicspath{{Chapter6/Figs/Raster/}{Chapter6/Figs/PDF/}{Chapter6/Figs/}}
\else
    \graphicspath{{Chapter6/Figs/Vector/}{Chapter6/Figs/}}
\fi

The VIDARR project is supposed to primarily focus on the measurement of $\Bar{\nu_e}$s from reactors. Whilst a prototype of the detector did measure $\Bar{\nu_e}$ data from 07-07-2014 to 25-02-2016 at the Wylfa reactor site (see figure \ref{fig:prototypeMeasumentFlux}) the results had a high noise level due to the condition of the electronics and container being sub-optimal. Both of which will be improved with better electronics and an improved container as described in section \ref{sec:DetectorConstruction}. In addition there were only 1793 channels that were instrumented out of a possible 2660 channels only $\sim$ 67\,\% of the detector. It is known that not only would the efficiency of the neutron data improve but the so would the quality of the containment of the gadolinium cascade (see figure \ref{fig:containment_comparison}). The upgrade seeks to address all of these issues increasing the overall signal to noise and increasing the number of signal events by $\sim$ 50\,\% from $\sim$ 200 (see figure \ref{fig:prototypeMeasumentFlux}) to $\sim$ 300. There are many experiments that are also looking at $\Bar{\nu_e}$s from reactors (see section \ref{sec:exisitingReactorMonitoringPrograms}). The number of these experiments suggest that this will be healthy and productive field for years to come.
\\\\In order to quantify the capabilities of the upgraded detector the GEANT4 simulation has been continuously worked on and improved in tandem with the upgrade. Taking into account the detector effects such as the MPPC response and dark noise ( see figure \ref{fig:fitting_of_non_peak_dark_noise}) as well as counting statistics (see figure \ref{fig:CoutingStats10}) and attenuation (see figure \ref{fig:attenuationPlot}) and quenching in the plastic (see section \ref{sec:geant4Simulation_quenchingLoss}). Unfortunately the on going pandemic caused by COVID-19 has restricted the supply of electronics and as such an error made by the producer of our electronics has been more time consuming to fix than expected. As a result there has been much focus on improving the simulation such as improving the Gd cascade (see section \ref{sec:geant4Simulation_gdCascade}) and testing machine learning techniques on that generated data (see section \ref{sec:MachineLearningTrigger}). But the biggest change to this thesis has been the large focus on cosmic $\mu$ tomography as opposed to analysing pure $\Bar{\nu_e}$ data.
\\\\Cosmic $\mu$ tomography at the Wylfa reactor site which is covered in chapter  \ref{chp:cosmicMuonTomography} and section \ref{sec:ReactorShadowMethodology} specifically shows how complicated the process can be. However, by utilising GEANT4 it is possible to get possible outlines of the buildings which is observed in figure \ref{fig:simulatedTrackerRecon} and these outlines can then be placed on top measured data seen in figure \ref{fig:measuredTrackerRecon}. These figures show how the buildings at the Wylfa reactor site project onto the VIDARR prototype and how well this matches data. This even allows for the resolution of the near reactor core or at least its containment/shielding. This was achieved by taking the ``transmission'' of how many cosmic $\mu$ pass through the buildings at the reactor site similar to the  MU-RAY collaboration's analysis of mt.Vesuvius \cite{Ambrosino_2014}.    
\\\\The analysis required a custom coded tracker built around the data from the prototype detector and simulated data to prevent biasing (see section \ref{sec:SimulationOfCosmics}). This tracker uses a simplex minimiser from the gnu scientific library \cite{galassi2002gnu} which forms the basis for most of the fitting and has no visible biasing associated with it (see figures \ref{fig:wylfaSideABHits}, \ref{fig:liverpoolSideABHits}). This tracker will be used both for cosmic $\mu$ tomography (which includes showers) and calibration (which excludes showers). 
\\\\Cosmic $\mu$ tomography should augment the VIDARR detector's capabilities as now it is possible for the detector to determine weather or not it has been moved. This is important as this will impact the measurements of $\Bar{\nu_e}$ and so will counter unscrupulous reactor sites from moving the detector in an attempt to give false readings. With the detector upgrade the VIDARR detector should be able to give more accurate $\Bar{\nu_e}$ readings and $\mu$ readings due to the increase in mass and improvements in electronics. 