% ************************** Thesis Abstract *****************************
% Use `abstract' as an option in the document class to print only the titlepage and the abstract.
\begin{abstract}

\normalsize This thesis will focus on the simulation and analysis of the Verification Instrument for the Direct Assay of Radiation at Range (VIDARR). It will cover the aims of the project and the technology used in its design. This is especially relevant as the detector is being upgraded, the process of which will be covered in this thesis as well. In order to quantify this upgrade, a GEANT4-based simulation has been created which includes the improvements to the electronics made in the upgrade as well as the physical effects of quenching and attenuation. In addition, the DICEBOX package is implemented using data from the DANCE calorimeter has been added to the simulation to increase fidelity of the gadolinium cascade which serves as VIDARR's trigger signal. A machine learning technique, an SVM, was used to separate out the generated Gd cascade from conservative noise with 75\,\% efficiency and 92\,\% purity. This trigger logic will be implemented using FPGA boards moving forwards. Finally, data from the prototype deployment at the Wylfa nuclear plant is analysed using a custom built cosmic $\mu$ Tracker. By analysing the azimuthal ($\phi$) and polar angle ($\theta$) determined by this tracker it is possible for VIDARR to image its surroundings. This is especially true when used in combination with the aforementioned simulation, utilising the CRY $\theta$ distribution \cite{ieee_cry_2007}, which allows for the resolution of many different buildings at the reactor site including height inferences and resolving the reactor core / reactor core shielding with as little as $\sim$ 3 hours of live time data. This will prove useful in aiding VIDARR's goal of reactor monitoring as it guards against the detector being moved. This also serves as a proof-of-concept of VIDARR as  an imaging tool and this capability will be incorporated into future deployment plans.\\
\providecommand{\keywordsAbs}[1]{\textbf{\textit{Keywords:}} #1} %Keywords command has to be supplied manually
\keywordsAbs{Monte Carlo, GEANT4, Anti-neutrino, Support Vector Machine (SVM), cosmic $\mu$ tomography, minimiser}

\end{abstract}
