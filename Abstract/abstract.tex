% ************************** Thesis Abstract *****************************
% Use `abstract' as an option in the document class to print only the titlepage and the abstract.
\begin{abstract}

\normalsize This thesis will focus on the simulation and analysis of the Verification Instrument for the Direct Assay of Radiation at Range (VIDARR). It will cover the aims of the project and the technology used in its construction. This is especially relevant as the detector is being upgraded, the process of which will be covered in this thesis as well. In order to help quantify this upgrade, a simulation has been created which includes the improvements to the electronics made in the upgrade as well as the physical effects of quenching and attenuation. In addition, a Dicebox based on the DANCE calorimeter has been added to the simulation to increase fidelity of the gadolinium cascade which serves as VIDARR's trigger signal. Finally data from the RMon prototype deployment at Wylfa is analysed using a custom built cosmic $\mu$ Tracker. By analysing the azimuthal ($\phi$) and polar angle ($\theta$) produced from this tracker it is possible for RMon or VIDARR to image their surroundings. This is especially true when used in combination with the aforementioned simulation which allows for the resolution of many different buildings at the reactor site. This includes height inferences and resolving the reactor core with as little as $\sim$ 3 hours of live time data. This should prove useful in aiding VIDARR's goal of reactor monitoring as it guards against the detector being moved. This also demonstrates the performance of VIDARR and RMon as imaging tools.\\
\providecommand{\keywordsAbs}[1]{\textbf{\textit{Keywords:}} #1} %Keywords command has to be supplied manually
\keywordsAbs{Monte Carlo, GEANT4, High performance computing (HPC), Anti-neutrino, Support Vector Machine (SVM), cosmic $\mu$ tomography}

\end{abstract}
