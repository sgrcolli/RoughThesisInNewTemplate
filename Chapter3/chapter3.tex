%!TEX root = ../thesis.tex
%*******************************************************************************
%****************************** Third Chapter **********************************
%*******************************************************************************
\chapter{The Upgraded Detector} \label{Chp:TheUpgradedDetector}

% **************************** Define Graphics Path **************************
\ifpdf
    \graphicspath{{Chapter3/Figs/Raster/}{Chapter3/Figs/PDF/}{Chapter3/Figs/}}
\else
    \graphicspath{{Chapter3/Figs/Vector/}{Chapter3/Figs/}}
\fi

\section{The Upgraded Detector}\label{sec_theUpgradedDetector}
Please note that in the following section of text the term upgraded and VIDARR detector are used interchangeably. The upgraded detector will have 21 more layers than the previous detector going from 49 to 70 layers and the 3 missing columns in side a are also now instrumented. This means the number of channels have increased from 1793 to 2660 which results in an increase of mass by $\sim$ 50\,\%, thus improving the fiducial volume of the detector. This means that more energy from the gadolinium cascade is contained within the detector thus allowing for more effective noise reduction when triggering. Therefore the increase in layers will allow for a higher efficiency of neutron capture. The increase in layers will not yield a significant increase in positron efficiency as positrons are effectively contained to a $\sim$ 99\,\% level with both 1793 channels and 2660 channels as they are contained within 1-2 bars.
\\\\The electronics have also been improved significantly compared to the original detector. The original electronics were carried over from the T2K Ecal, they were the first generation of the technology with relatively high noise rates compared to the current generation used in the VIDARR detector. The energy resolution of the electronics have been greatly increased \hl{(any quantifiable numbers for this stuff?)}. And the noise rates for the original electronics were also more susceptible to changes in temperature than the electronics in the upgraded detector. In addition the upgraded detector will also have field-programmable gate array (FPGA) boards which connect with analogue boards which in turn connect to the MPPCs. The use of FPGA boards allow for more complex trigger functions to be used with up to two thresholds to be used and allows for the summed energy and the number of bars to be hit to be used for trigger discrimination. Which should allow for better signal to noise discrimination than the original detector.
\\\\A basic cut investigation used a form of machine learning called a support vector machine (SVM) to determine the best cut and which dimensions gave the best separation. The cut was dominated by the number of bars hit at the lower threshold of 0.1\,MeV, the most accurate cut would have been utilising the number of bars hit above 0.1\,MeV and the summed energy above the 0.1\,MeV threshold. However due to the structure of the FPGA boards and their programming it was more prudent to use the number of bars hit above both the 0.1\,MeV thresholds and 0.5\,MeV threshold. The cost in classifier accuracy was minimal and it allowed for faster development of the FPGA firmware. 
\\\\To lessen temperature fluctuations in the VIDARR detector the cooling in and around the detector module has been greatly increased from the prototype. The original detector had six radiator fins on 2 sides of the detector which were primarily aimed at cooling the TFBs on each fin. The upgraded VIDARR detector will also have these fins which will help cool the new boards but on the same 2 sides as the fins there will also be two new radiators behind the fins which run the width and height of a side. Greatly increasing the speed that heat will dissipate inside the detector. This will allow for a more consistent temperature  thus reducing dark noise even further from the MPPCs. 





