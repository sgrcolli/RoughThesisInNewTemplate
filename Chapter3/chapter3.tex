%!TEX root = ../thesis.tex
%*******************************************************************************
%****************************** Third Chapter **********************************
%*******************************************************************************

% **************************** Define Graphics Path **************************
\ifpdf
    \graphicspath{{Chapter3/Figs/Raster/}{Chapter3/Figs/PDF/}{Chapter3/Figs/}}
\else
    \graphicspath{{Chapter3/Figs/Vector/}{Chapter3/Figs/}}
\fi

%*******************************************************************************
%****************************** Second Chapter *********************************
%*******************************************************************************

\chapter{The RMon and VIDARR Detectors}\label{Chp:ThePrototypeDetector}
This chapter will discuss the prototype VIDARR detector (RMon) and the on going upgrade of that detector into VIDARR proper. In addition, the initial basis for both detectors the T2K ND 280 (near detector at 280\,m) Electromagnetic Calorimeter (ECal) will also be covered. As well as a brief mentioning of the background at reactor sites. The upgrade progressed steadily until December 2019. Despite the ongoing uncertainty the detector has had a $\sim$ 50\,\% increase in mass and new cooling and containment. At time of writing the electronics are currently awaiting upgrade. 

\section{Technology And Results Of RMon}
%This thesis will cover consider RMon and the upgraded detector VIDARR. RMon the original prototype detector was re-purposed technology from the T2K ND280 (Near Detector (at) 280\,m) Electromagnetic Calorimeter (ECal) \cite{Allan_2013}. The VIDARR detector uses the same detection media but upgraded electronics and containment which will be referred to as the VIDARR detector. This section will focus on the RMon detector.
This section will focus on the RMon detector and so a look at the basis for both RMon, the ND280 ECal, is required. The ND280 detector located at JPARC for the T2K experiment was particularly used to study the $\nu_\mu$ beam used for neutrino oscillation measurements. The detector consists of central sub-detectors comprising of time guarded scintillator bus fine-grained detectors (FGDs) and time projection chambers (TPCs) using lead as the interaction medium surrounded by ECals to stop and measure particles leaving the central detector (see figure \ref{fig:nd280Fig}). The ECals used lead sheets as a conversion material as well as a secondary target. 
%\\\\As it is the basis for the other detectors a quick overview of the T2K ND280 ECal will be required. The ND280 is a series of detectors from the neutrino oscillation experiment T2K which relies on a $\nu_\mu$ beam entering the detector shown in figure \ref{fig:nd280Fig}. This detector was comprised of several different detectors including time projection chambers (TPCs) and fine-grained detectors (FGDs) and electromagnetic calorimeters (ECals) \cite{Allan_2013}. The ECals are of particular interest as they are the basis for the RMon and VIDARR detector due to their robust design inert nature and high precision.  

\begin{figure}[!h]
 \centering
 \includegraphics[width=\linewidth/2]{Chapter2/Figs/Raster/ND280Fig.png} 
 \captionof{figure}{Diagram of the ND280 detector the RMon detector is based on the Barrel ECAL. From \cite{Allan_2013}} %~can be used as a kind of place holder in latex
 \label{fig:nd280Fig}
\end{figure}

The T2K ECals were made from plastic scintillating bars measuring $4\,\textrm{cm}\times1\,\textrm{cm}$ with varying lengths arranged in alternating layers at 90 degrees to each other (see figure \ref{fig:vidarrDiagram}) \cite{Allan_2013}. Wavelength shifting (WLS) fibres were placed in the centre of the scintillating bars to measure light collection which shift the wavelength from blue to green to increase the quantum efficiency of the sensors \cite{Allan_2013}. These wavelength shifting fibres are then connected to multi-pixel photon counters (MPPCs) which produce pulses of charge $\sim$ proportional to the number of photons detected. This in turn is related to the energy deposited in the bar. This signal is then read in by Trip-T front-end electronic boards (TFBs) which integrates the charged received into an array of 23 time buckets of programmable width (480\,ns for ND280, 1.5\,$\mu$s for RMon). 
\\\\However, a crucial difference between the ECals of the ND280 and the RMon detector is that the lead sheets have been replaced with a layer of Mylar coated with gadolinium sulphate so that neutrons produced in IBD (equation \ref{inverse_beta_decay}) are captured. RMon was also designed to fit inside of a shipping container each bar having a length of 152\,cm and the whole detector measuring $152\,\textrm{cm}\times152\,\textrm{cm}$ with 49 layers of plastic scintillating bars giving a height of 49\,cm and a total of 1862 bars. The electronic systems were also adapted from the T2K ECal. They were configured such that they were triggered from a neutron-induced gadolinium cascade from IBD by looking for a large number of hits in the trigger cycle. Of the 23 cycles 0 -- 17 are considered "prompt" and cycle 18 is the trigger cycle. Cycles 19 -- 22 were left as control so that they could be compared in case time-dependent issues arose from the altered system. As will be discussed later in chapter \ref{chp:cosmicMuonTomography} cycles 19 -- 22 were used to diagnose a time dependant issue with the cosmic $\mu$ data set.
\\\\$\Bar{\nu_e}$s are detected by their absorption by a proton and decay into a $e^+$ and a neutron (see section \ref{subSec:IBD}). The e$^+$ and subsequent annihilation $\gamma$ rays leave short tracks in the detector. The annihilation particles may be used to distinguish the e$^+$ from the $\beta$ decay e$^-$ backgrounds. In an IBD event the neutron is absorbed by the Gadolinium sheets in the detector which then causes an 8\,MeV $\gamma$ cascade which will be expanded upon in section \ref{sec:GEANT4Simulation_modellingGadolinium}. Gadolinium is one of the most preferred neutron capture agents due to the higher energy $\gamma$-rays and its extremely high capture cross-section of 48890\,b \cite{gadFoilsThermalNeut}. 

\begin{figure}[!h]
\centering
  \centering
  \includegraphics[width=0.6\linewidth]{Chapter2/Figs/Raster/VIDARR_diagram.jpeg}
  \captionof{figure}{A cutaway diagram showing the structure of the ND280 ECal, RMon, and VIDARR detectors. The segments are 4\,cm wide $\times$ 152\,cm long $\times$ 1\,cm tall. The Gadolinium sheets in-between the RMon/VIDARR segments are not shown. From \cite{GeorgeHoltDiagram}.}
  \label{fig:vidarrDiagram}
\end{figure}

The RMon detector was deployed at the Wylfa power station in Anglesey Wales for an 18-month period. This run proved successful in measuring the power from the reactor to within good agreement of the measured reactor flux. Figure \ref{fig:prototypeMeasumentFlux} compares the measured $\bar{\nu_e}$ flux with the reported power output and shows how good the agreement is. A measured anti-neutrino rate of 172.1 $\pm$ 4.6 candidates per day is observed when the reactor is off and 203.7 $\pm$ 19.6 when the reactor is on \cite{Carroll_2018}. These results line up with the expectation from the SONGS1 deployment (figure \ref{subFig:reactorPowerSongsS1}) where a increase in $\bar{\nu_e}$ candidates correlates with an increase in reactor power. %Unfortunately due to cooling issues with the RMon detector the reactor shutdown was not observed. This is one of the main motivating factors behind the upgrade of the detector as the first generation MPPCs and repurposed electronics were susceptible to high levels of noise if the temperature was not carefully controlled. 
\begin{figure}[!h]
 \centering
 \includegraphics[width=0.7\linewidth]{Chapter2/Figs/Raster/prototypeMeasureOnFig.png} 
 \captionof{figure}{Measured anti-neutrino flux compared to the power generation from the Wylfa power station. The increase in $\bar{\nu_e}$ candidates correlates with an increase in power generation. These results demonstrate the success of the RMon deployment and technology. From \cite{Carroll_2018}} %~can be used as a kind of place holder in latex
 \label{fig:prototypeMeasumentFlux}
\end{figure}

Considering the re-purposed nature and rapid development cycle of RMon the deployment at Wylfa and subsequent neutrino measurement (see figure \ref{fig:prototypeMeasumentFlux}) is a good demonstration of the detection media and experimental setup. However, whilst these results are highly encouraging the response measurement can still be improved, for example the SONGS1 measurement is more accurate (see figure \ref{fig:reactorPowerAndRefuelingSongsS1}). In addition, the PANDA collaboration has been able to measure a neutrino spectrum the from Ohi power plant at a distance of $\sim$ 50\,m \cite{IIRIE_Panda_2021} (see figure \ref{subFig:Panda_spectrumOfIbdCandidates}). With an upgraded version of the RMon detector it should be possible to achieve similar results to SONGS1 and PANDA and current modelling is on going to assess the nuclear waste measuring potential of VIDARR. 

\section{Studies Of Background At Reactor Sites}
Backgrounds at reactor sites are important to quantify as it helps to guide shielding requirements. The results from SONGS1, PANDA, and RMon all show low count rates per day of $\sim$ 100 per ton. These low count rates necessitate a good understanding of background. The PROSPECT experiment has done a  study looking into the background at two research reactor locations, the National Bureau of Standards Reactor (NBSR) at NIST and the High Flux Isotope Reactor (HFIR) at ORNL \cite{Ashenfelter_2016}. Another site was considered by the PROSPECT collaboration, the ATR at INL but the increased altitude of that site leads to a significantly higher cosmogenic neutron flux \cite{Ashenfelter_2016}. An example of how reactors affect the background can be seen in figure \ref{fig:Prospect_NSBR_gammaSpec} which shows how the $\gamma$ spectrum varies when the NBSR is on and off, it shows how the spectrum between 3\,MeV -- 9\,MeV is mostly dominated by reactor noise. 
\\\\Another component is neutron measurements caused by the reactor which is shown in figure \ref{fig:prospectNeutronMap} for thermal neutrons which are an issue for $\bar{\nu_e}$ detectors but should be somewhat negated through robust e$^+$ identification and accurate double coincident measurements. Another more important background is fast neutrons because they cause a false double coincident signal by potentially interacting with protons and then thermalising and being absorbed. The measurement of cosmogenic neutron background has been done by JEDEC in 2006 \cite{JEDEC_2006} as well as PROSPECT in figure \ref{fig:Prospect_HFIR_NBSR_nearFarPlots} which shows similar rates at different locations this is unsurprising as this is largely a function of altitude more than any other factor \cite{Ashenfelter_2016}. \\\\However, a type of background that is easy to mitigate (for RMon and VIDARR) but also extremely useful are cosmic $\mu$. Cosmic $\mu$ have a large number of events of $\sim$ 119\,s$^{-1}$m$^{-2}$ according to the CRY library \cite{ieee_cry_2007}. Cosmic $\mu$ form the later half of this thesis (chapters \ref{chp:DataAnalysisTechniques}, \ref{chp:cosmicMuTelescopes}, \ref{chp:cosmicMuonTomography}) and they will be expanded upon then. They are noise for $\bar{\nu}$ events as they hit a large number of channels thus setting off the neutron triggers these experiments rely on. Conversely their high energy and highly penetrating nature means only a simple detector is required to quantify their rat, PROSPECT for example used the cosmic $\mu$ telescope seen in figure \ref{fig:Prospect_MuonPaddels}. The segmentation of RMon and VIDARR allows them to reject cosmic $\mu$ noise very effectively. This is because cosmic $\mu$ can be reasonably approximated as straight line tracks which are more easily identified with finer segmentation. % (also see figure \ref{fig:CRY_rates}). These particles can be detected by very simple detector requiring only a series of paddles as seen in figure \ref{fig:Prospect_MuonPaddels}. This cosmic $\mu$ background is useful as they are highly penetrating particles tracks that can be approximated to be straight lines (later discussed in chapter \ref{chp:cosmicMuonTomography}). The range of angles incident cosmic $\mu$ posses make them useful for tomographic purposes as will be discussed in chapter \ref{chp:cosmicMuTelescopes}. %% The angles of incidence of cosmic $\mu$ can have many useful applications and as such these cosmic $\mu$ should not be excluded as mere noise as will be shown in section \ref{sec:cosmicMuTelescopes}.


\begin{figure}[!h]
\centering
\begin{minipage}{.45\textwidth}
  \centering
  \includegraphics[width=\linewidth]{Chapter2/Figs/Raster/Prospect_NSBR_gammaSpec.png}
  \captionof{figure}{Example HPGe $\gamma$-ray spectra taken with the NBSR on and off. Prominent lines, and associated escape peaks and Compton continua, are evident. From \cite{Ashenfelter_2016} (table 2 in \cite{Ashenfelter_2016} shows line sources).} 
  \label{fig:Prospect_NSBR_gammaSpec}
  \vspace{1.912cm} %1 line = 0.478cm % 2 lines = 0.956cm % 3 lines= 1.434cm % 4 lines = 1.912cm % 5 lines = 2.39cm
\end{minipage}%
\qquad
\begin{minipage}{.45\textwidth}
  \centering
  \includegraphics[width=\linewidth]{Chapter2/Figs/Raster/prospectNeutronMap.png} 
  \captionof{figure}{A pictorial representation of neutron dose rates (measured in nSv/h) and thermal neutron rates in italics (cm$^{-2}$ s${^-1}$) at the HFIR near location roughly 15\,cm (z = 0.15) above the floor. Measurements are plotted on a one meter square grid referenced to the reactor wall (x = 0) and the smallest baseline (y = 0). The reactor core is centred at (x,y,z) = (-4.06,0,-3.85). From \cite{Ashenfelter_2016}.}
  \label{fig:prospectNeutronMap}
\end{minipage}
\end{figure}

% \begin{figure}[!h]
%  \centering
%  \includegraphics[width=0.7\linewidth]{Chapter2/Figs/Raster/JDEC_neutronSpec.png}
%  \captionof{figure}{The JEDEC standard fast neutron spectrum recorded at sea
% level in New York. From \cite{JEDEC_2006}. } 
%  \label{fig:JDEC_neutronSpec}
% \end{figure}

\begin{figure}[!h]
\centering
\begin{subfigure}{.5\textwidth}
  \centering
  \includegraphics[width=\linewidth]{Chapter2/Figs/Raster/Prospect_HFIR_nearFarPlot.png}
  \captionsetup{width=.9\linewidth}
  \caption{}
  \label{subFig:Prospect_HFIR_nearFarPlot}
\end{subfigure}%
\begin{subfigure}{.5\textwidth}
  \centering
\includegraphics[width=\linewidth]{Chapter2/Figs/Raster/Prospect_NBSR_farPlot.png}
  \captionsetup{width=.9\linewidth}
  \caption{}
  \label{subFig:Prospect_NBSR_farPlot}
\end{subfigure}
\caption{The cosmogenic neutron-induced energy spectrum was recorded at the (a) HFIR near and far locations and (b) NBSR far location. From \cite{Ashenfelter_2016}.}
\label{fig:Prospect_HFIR_NBSR_nearFarPlots}
\end{figure}

\begin{figure}[!h]
 \centering
 \includegraphics[width=0.7\linewidth]{Chapter2/Figs/Raster/Prospect_MuonPaddels.png}
 \captionof{figure}{The angular acceptances for the $\mu$ telescope instrument
used at all sites is determined by the coincidence requirement enforced
between the 4 plastic scintillator paddles. These detectors were deployed by the PROSPECT experiment at both the NBSR and HFIR sites. From \cite{Ashenfelter_2016}.} 
 \label{fig:Prospect_MuonPaddels}
\end{figure}

%\clearpage
%white space
\vspace{3cm}

\section{The VIDARR Detector Upgrade}\label{sec:theUpgradedDetector}
The deployment of RMon demonstrated that the choice of detector media (scintillating bars Gd doped sheets and wavelength recorded by WLS fibres) worked well. However, the detector would benefit from more mass, higher efficiency sensors and dedicated electronics. The upgraded detector will have 21 more layers than the RMon detector going from 49 to 70 layers and the 3 missing columns on side A are also now instrumented (see figure \ref{fig:detectorUpgradedmassOutlined}). This means the number of channels has increased from 1793 to 2660 which results in a mass increase of $\sim$ 50\,\%, thus improving the fiducial volume of the detector. The increase in layers will increase the average containment of the 8\,MeV Gd cascade in the detector allowing for more effective background reduction when triggering. The MPPCs are also a new generation with 2$\times$ the efficiency going from RMon to VIDARR. %The increase in layers will not yield a significant increase in e$^+$ efficiency as e$^+$s are effectively contained to a $\sim$ 99\,\% level with both 1793 channels and 2660 channels as they are contained within 1-2 bars.

\begin{figure}[!h]
 \centering
 \includegraphics[width=0.5\linewidth]{Chapter3/Figs/detectorUpgradedmassOutlinedLabelled.png}
 \captionof{figure}{The upgraded detector mass (light blue) on top of the original mass (dark blue). The greyed out channels on the left hand side of side A were those left un-instrumented in the original deployment but are now instrumented in the upgrade.} 
 \label{fig:detectorUpgradedmassOutlined}
\end{figure}

The electronics have been improved significantly when compared to the original detector. The original electronics and sensors were taken from the T2K ECal. The MPPCs were the first generation of this technology with relatively low efficiency and high noise rates compared to the current generation used in the VIDARR detector. The energy resolution of the detector has been greatly improved by using more modern, higher efficiency MPPCs. In addition, the upgraded detector will have channel by channel trigger output to Field-Programmable Gate Array (FPGA) boards that connect with analogue boards in turn connecting to the MPPCs. The use of FPGA boards will allow for more complex trigger functions to be used and trigger on the detector as a whole. This will be achieved by looking at the summed energy and the number of bars hit past each threshold. As will be expanded upon later in section \ref{sec:MachineLearningTrigger} the improvement to energy resolution has a significant effect on the S/N ratio as well. During the analysis the lowest available threshold of $\sim$ 0.1\,MeV was the most effective in distinguishing signal from noise. This improvement is only possible due to the upgraded sensors and electronics. Both generations of MPPCs are highly temperature sensitive, and this did cause stability issues for the RMon detector. %The use of two thresholds as opposed to one on the RMon also proved useful for improving generated neutron efficiency. 
%\\\\A basic cut investigation used a form of machine learning called a support vector machine (SVM) to determine the best cut and which dimensions gave the best separation. The cut was dominated by the number of bars hit at the lower threshold of 0.1\,MeV, the most accurate cut would have been utilising the number of bars hit above 0.1\,MeV and the summed energy above the 0.1\,MeV threshold. However due to the structure of the FPGA boards and their programming it was more prudent to use the number of bars hit above both the 0.1\,MeV thresholds and 0.5\,MeV threshold. The cost in classifier accuracy was minimal and it allowed for faster development of the FPGA firmware. 
\\\\To lessen temperature fluctuations in the VIDARR detector the cooling in and around the detector module has been greatly increased from the RMon prototype. RMon had six radiator fins on 2 sides of the detector which were primarily aimed at cooling the TFBs on each fin (see figure \ref{fig:detCon002_OldTearAway}). The upgraded VIDARR detector will also have cooling fins which will take heat away from the new boards but on the same 2 sides as the fins, there will also be two new radiators behind the fins which run the width and height of a side. The new active cooling the radiators will provide cooling to the cavity housing to the MPPCs and shield them from heat generated by the active electronics. This will allow for a more consistent temperature and reduce dark noise from the MPPCs below the original detector's levels. In addition temperature monitoring has also been added to ensure the MPPCs are maintained at the correct temperature.

\section{Detector Construction}\label{sec:DetectorConstruction}
In December of 2018, the original detector was moved out of the shipping container which housed it (figure \ref{fig:detCon000_TakeOut1}). It was then placed inside an ISO class 7 cleanroom opened and disassembled (figure \ref{fig:detCon002_OldTearAway}). The RMon detector had much space that could be further utilised. The internal space of RMon's shell was filled with 1\,m long cables connecting the sensors to the electronics (see figure \ref{fig:detCon002_OldTearAway}) and polystyrene filler in lieu of the top 21 layers. RMon's shell would be kept for VIDARR but instead the space inside the shell would be more efficiently utilised allowing for more electronics, sensors, and scintillator. The new scintillator purchased from Fermilab was $\sim$ twice as long as required and so was cut (see figure \ref{subFig:detCon003bb_CuttingScint}) so its dimensions matched the old scintillating bars ($152\,\textrm{cm} \times 4\,\textrm{cm} \times 1\,\textrm{cm}$). Once this was done, the edges of the freshly cut scintillator were arranged and painted with TiO$_2$ paint (figure \ref{subFig:detCon005b_PaintingEnds}) to help reflect the light. The cut and painted scintillating bars were then placed inside the housing for the detector whilst it was in the cleanroom environment to minimise background/contamination (figure \ref{fig:detCon006_RonInCleanRoom}). The layers of scintillator have a white sheet of Mylar coated with Gadolinium sulphate in between them, the reflective white sheet can be seen in figure \ref{fig:detCon006_RonInCleanRoom}, the Gadolinium sulphate is necessary for capturing neutrons produced during IBD (equation \ref{inverse_beta_decay}). 

\begin{figure}[!h]
\centering
\begin{minipage}{.45\textwidth}
  \centering
  \includegraphics[width=\linewidth]{Chapter3/Figs/Raster/detCon002_OldTearAway.png}
  \captionof{figure}{A view of the electronics in the RMon detector. There is much available space between the boards and the detector itself. The cables are 100\,mm long.} 
  \label{fig:detCon002_OldTearAway}
\end{minipage}%
\qquad
\begin{minipage}{.45\textwidth}
  \centering
  \includegraphics[width=\linewidth]{Chapter3/Figs/Raster/detCon000_TakeOut1.png} 
  \captionof{figure}{The RMon detector being taken out of the original shipping container which was a standard cooled shipping container.}
  \label{fig:detCon000_TakeOut1}
  \vspace{0.478cm} %1 line = 0.478cm % 2 lines = 0.956cm % 3 lines= 1.434cm % 4 lines = 1.912cm % 5 lines = 2.39cm
\end{minipage}
\end{figure}

\begin{figure}[!h]
\centering
\begin{subfigure}{.5\textwidth}
  \centering
  \includegraphics[width=\linewidth]{Chapter3/Figs/Raster/detCon003bb_CuttingScint.png}
  \captionsetup{width=.9\linewidth}
  \caption{}
  \label{subFig:detCon003bb_CuttingScint}
\end{subfigure}%
\begin{subfigure}{.5\textwidth}
  \centering
  \includegraphics[width=\linewidth]{Chapter3/Figs/Raster/detCon005b_PaintingEnds.png}
  \captionsetup{width=.9\linewidth}
  \caption{}
  \label{subFig:detCon005b_PaintingEnds}
\end{subfigure}
\caption{Scintillator preparation for being placed inside the detector casing. The scintillator is being cut in (a) with arranged for painting the ends in (b).}
\label{fig:detCon_CuttingScint_PaintingEnds}
\end{figure}

\begin{figure}[!h]
\centering
\includegraphics[width=0.7\linewidth]{Chapter3/Figs/Raster/detCon006_RonInCleanRoom.png}
\captionof{figure}{The author (Ronald Collins) assembling the final layer of the VIDARR detector. The white sheet of Gadolinium Oxide in-between the layers is also visible.} 
\label{fig:detCon006_RonInCleanRoom}
\end{figure}

\begin{figure}[!h]
\centering
\begin{subfigure}{.5\textwidth}
  \centering
  \includegraphics[width=\linewidth]{Chapter3/Figs/Raster/detCon011b_RadiatorConstruction.png}
  \captionsetup{width=.9\linewidth}
  \caption{}
  \label{subFig:detCon011b_RadiatorConstruction}
\end{subfigure}%
\begin{subfigure}{.5\textwidth}
  \centering
  \includegraphics[width=\linewidth]{Chapter3/Figs/Raster/detCon012b_RadiatorPiping.png}
  \captionsetup{width=.9\linewidth}
  \caption{}
  \label{subFig:detCon012b_RadiatorPiping}
\end{subfigure}
\caption{The construction of the new radiator has more active cooling and a larger surface area than the original radiator. The radiator is being cut in (a) and the piping is inserted in (b).}
\label{fig:detCon_RadiatorConstruction_RadiatorPiping}
\end{figure}

The upgraded electronics for VIDARR draw far more power and require substantially more cooling than the RMon electronics, and major improvements were made to cool the temperature sensitive MPPCs and thermally isolate them from the electronics. The original detector had much space between the electronic boards and the MPPCs with the electronic boards placed directly on cooling fins (see in figure \ref{fig:detCon002_OldTearAway}). In addition to the cooling fins seen in figure \ref{fig:detCon002_OldTearAway} large radiators have now been added in-between the cooling fins and the MPPCs for each side. These new radiators are comprised of large stainless steel plates which were cut shown in figure \ref{subFig:detCon011b_RadiatorConstruction}. Then copper piping was placed through the sections seen in figure \ref{subFig:detCon012b_RadiatorPiping}. This copper piping will face the MPPCs and the detector scintillator whilst the radiator fins will be attached behind the radiator away from the MPPCs and behind a 1\,cm thick (in X) layer of insulation. As in RMon, the electronic boards will be placed on the fins similar to figure \ref{fig:detCon002_OldTearAway}. This should improve the cooling in the detector significantly and provide the isolation between the electronics boards and the scintillator and the MPPCs when compared to the RMon detector. 

% Once the scintillator is in the detector casing several individual components need to be assembled so that the light emitted by the scintillator can then be analysed. A small amount of light emitted by the scintillator upon particle interaction is trapped by the WLS fibres . These fibres are threaded through the centre of the scintillator and capture light and shift it to green light seen in figure \ref{subFig:detCon013b_WlsFibres}. These WLS fibres then have 3D printed connectors glued to the ends of them as seen in figure \ref{subFig:detCon014b_WlsWithEnds}. 3D printing was done as more vacuum moulded parts couldn't be obtained from the original supply chain which RMon used. The 3D printing was done by a Formlabs Form2 resin printer, the components were similar in quality but slightly more brittle and much easier to obtain.

Before insertion into the scintillator bars, the wavelength shifting fibres needed to be connected to aid the transport of wavelength shifted scintillation light to the MPPCs. Male connectors were glued to one end of each fibre, these parts were originally injection moulded, however due to problems with the supply chain the majority of the new stock were 3D printed using a Formlabs Form2 resin printer. The ease of manufacture outweighed the disadvantage of slightly more brittle components. The fibres with and without connectors attached are shown in figure \ref{fig:detCon_WlsFibres_WlsWithEnds}. Then the MPPCs connectors are 3D printed seen both with the support struts in figure \ref{subFig:detCon015b_3dPrintedHolders} and without in figure \ref{subFig:detCon016b_3dPrintedFreed}. The holders for the MPPCs were also 3D printed. These connect with the WLS fibres, aligning the output of the fibre onto the MPPC sensor as well as providing a mounting point for mini-pcbs which electronically connect to the terminals of the MPPCs. A sheet of which can be seen in figure \ref{subFig:detCon008b_PlacingPcbs}. These PCBs have to have a micro-coaxial cable connector soldered on to them and have through-hole connections for the MPPCs. A finished example of one of these PCBs can be seen in figure \ref{subFig:detCon009b_SoloPcb}. The holders, PCBs, and MPPCs all connect together as shown in figure \ref{fig:detCon017b_HoldersWithParts}. 

\begin{figure}[!h]
\centering
\begin{subfigure}{.5\textwidth}
  \centering
  \includegraphics[width=\linewidth]{Chapter3/Figs/Raster/detCon013b_WlsFibres.png}
  \captionsetup{width=.9\linewidth}
  \caption{}
  \label{subFig:detCon013b_WlsFibres}
\end{subfigure}%
\begin{subfigure}{.5\textwidth}
  \centering
  \includegraphics[width=\linewidth]{Chapter3/Figs/Raster/detCon014b_WlsWithEnds.png}
  \captionsetup{width=.9\linewidth}
  \caption{}
  \label{subFig:detCon014b_WlsWithEnds}
\end{subfigure}
\caption{a) The WLS fibres b) with connectors attached. The green light emitted at the ends of the fibre is ambient light wavelength shifted by the fibres.} %The WLS fibres will funnel light from the scintillator to the MPPCs. They are assembled on cardboard and have 3d printed ends glued onto them. In (a) the WLS fibres are prepared for their ends by lying on cardboard. In (b) the ends have been glued on.
\label{fig:detCon_WlsFibres_WlsWithEnds}
\end{figure}

\begin{figure}[!h]
\centering
\begin{subfigure}{.5\textwidth}
  \centering
  \includegraphics[width=\linewidth]{Chapter3/Figs/Raster/detCon015b_3dPrintedHolders.png}
  \captionsetup{width=.9\linewidth}
  \caption{}
  \label{subFig:detCon015b_3dPrintedHolders}
\end{subfigure}%
\begin{subfigure}{.5\textwidth}
  \centering
  \includegraphics[width=\linewidth]{Chapter3/Figs/Raster/detCon016b_3dPrintedFreed.png}
  \captionsetup{width=.9\linewidth}
  \caption{}
  \label{subFig:detCon016b_3dPrintedFreed}
\end{subfigure}
\caption{The holders for the MPPCs and the PCBs. Holders for the additional channels needed to be 3D printed as more from the original supply chain could not be procured. In (a) the holders have the support struts attached and in (b) they have been removed.}
\label{fig:detCon_3dPrintedHolders_3dPrintedFreed}
\end{figure}

\begin{figure}[!h]
\centering
\begin{subfigure}{.5\textwidth}
  \centering
  \includegraphics[width=\linewidth]{Chapter3/Figs/Raster/detCon008b_PlacingPcbs.png}
  \captionsetup{width=.9\linewidth}
  \caption{}
  \label{subFig:detCon008b_PlacingPcbs}
\end{subfigure}%
\begin{subfigure}{.5\textwidth}
  \centering
  \includegraphics[width=\linewidth]{Chapter3/Figs/Raster/detCon009b_SoloPcb.png}
  \captionsetup{width=.9\linewidth}
  \caption{}
  \label{subFig:detCon009b_SoloPcb}
\end{subfigure}
\caption{The assembly of the PCBs which attach to the MPPCs through pins. (a) shows the sheet which had them manufactured and (b) shows a completed component. The connector at the top of the PCBs is where the cables are attached. }
\label{fig:detCon_PlacingPcbs_SoloPcb}
\end{figure}

Each of the micro-coaxial cables that connect the MPPCs to the analogue boards were labelled with heat shrink labels to ensure the labels were secure. The label syntax is side-row-column with the bottom left of each side being the coordinate for (0,0). An example of this syntax would be B-10-37 indicating side B row 10 and column 37. These labels were then slid onto the micro-coaxial cables 8\,cm from the end that connects to the analogue boards and heat shrunk. These cables were then connected to the holders where the PCB connector seen in figure \ref{subFig:detCon009b_SoloPcb} connects firmly to the cable. The holder was also put into a sheath so that was held firmly in place once inside the detector, as seen in figure \ref{fig:detCon023b_HoldersConnectedZoom}.

\begin{figure}[!h]
\centering
\begin{minipage}{.45\textwidth}
  \centering
  \includegraphics[width=\linewidth]{Chapter3/Figs/Raster/detCon017b_HoldersWithParts.png}
  \captionof{figure}{Holder next to MPPC and PCB (Top) and holder with assembled components (Bottom). Note the MPPC in the top right is actually reversed from its correct orientation, the prongs of the MPPC go through the PCB board pins.} 
  \label{fig:detCon017b_HoldersWithParts}
\end{minipage}%
\qquad
\begin{minipage}{.45\textwidth}
  \centering
  \includegraphics[width=\linewidth]{Chapter3/Figs/Raster/detCon023b_HoldersConnectedZoom.png} 
  \captionof{figure}{The holder is connected to a cable and put inside a sheath.}
  \label{fig:detCon023b_HoldersConnectedZoom}
  \vspace{1.912cm} %1 line = 0.478cm % 2 lines = 0.956cm % 3 lines= 1.434cm % 4 lines = 1.912cm % 5 lines = 2.39cm
\end{minipage}
\end{figure}

% \begin{figure}[!h]
% \centering
% \begin{subfigure}{.5\textwidth}
%   \centering
%   \includegraphics[width=\linewidth]{Chapter3/Figs/Raster/detCon018b_HeatLabelsPrinted.png}
%   \captionsetup{width=.9\linewidth}
%   \caption{Heat shrink labels being printed out.}
%   \label{subFig:detCon018b_HeatLabelsPrinted}
% \end{subfigure}%
% \begin{subfigure}{.5\textwidth}
%   \centering
%   \includegraphics[width=\linewidth]{Chapter3/Figs/Raster/detCon019b_CutLabels.png}
%   \captionsetup{width=.9\linewidth}
%   \caption{Individual labels for specific cables.}
%   \label{subFig:detCon019b_CutLabels}
% \end{subfigure}
% \caption{Heat-shrink labels are used to identify cables. Labels are printed out in the form side-Row-Column where the bottom left is (0,0) }
% \label{fig:detCon_HeatLabelsPrinted_CutLabels}
% \end{figure}

% \begin{figure}[!h]
% \centering
% \begin{subfigure}{.5\textwidth}
%   \centering
%   \includegraphics[width=\linewidth]{Chapter3/Figs/Raster/detCon020b_LabelsLoose.png}
%   \captionsetup{width=.9\linewidth}
%   \caption{}
%   \label{subFig:detCon020b_LabelsLoose}
% \end{subfigure}%
% \begin{subfigure}{.5\textwidth}
%   \centering
%   \includegraphics[width=\linewidth]{Chapter3/Figs/Raster/detCon021b_LabelsHeated.png}
%   \captionsetup{width=.9\linewidth}
%   \caption{}
%   \label{subFig:detCon021b_LabelsHeated}
% \end{subfigure}
% \caption{Labels are threaded through at 8\,cm from the end of the cables as seen in (a) then a heat gun is used on the cables to shrink the labels and they are bunched as seen in (b).}
% \label{fig:detCon_LabelsLoose_LabelsHeated}
% \end{figure}

All of the MPPC holders in their sheaths with connected cables are then attached to the WLS fibres and their 3D printed connectors shown in figure \ref{subFig:detCon014b_WlsWithEnds}. There are two distinct sheath types: grey made via injection moulding and orange sheaths which were 3d printed. These sheaths protect the rest of the components and can be seen clearly in figure \ref{fig:detCon_HaningOffRadiator_RadiatorTopDown} as they protrude from the detector casing. The radiator was slowly moved into position as cables were threaded through the insulation as can be seen in figure \ref{subFig:detCon026b_HaningOffRadiator}. Once the radiator was moved into position the space between the radiator and sheaths was checked to ensure that no undue pressure was being applied to the sheaths (figure \ref{subFig:detCon028b_RadiatorTopDown}). Once the radiator was in position the cooling fins were  attached as (figure \ref{fig:detCon030b_RadiatorWithFins}). X shaped base-plate board holders are then screwed into cooling fins (figure \ref{fig:detCon030b_RadiatorWithFins}) with thermal paste between the fins and and base plates to counteract any lack of contact between the base plates and the cooling fins. An example of how the analogue processing boards (APB) are attached to the cooling fins can be seen in figure \ref{fig:detCon032_ConnectedBoard}. Which shows how the cables are connected to the APB and aligned through an orange cable comb. To aid cooling a $5\,\textrm{cm}\times5\,\textrm{cm}$ thermal pad will be placed between the base-plate and the APB. Then ADC mezzanine boards (AMB) will then connect to the APBs. The AMB connected to the top of the APB will be air cooled, with heat generating components (ADCs, FPGA) mounted with heat sinks.

\begin{figure}[!h]
\centering
\begin{subfigure}{.5\textwidth}
  \centering
  \includegraphics[width=\linewidth]{Chapter3/Figs/Raster/detCon026b_HaningOffRadiator.png}
  \captionsetup{width=.9\linewidth}
  \caption{}
  \label{subFig:detCon026b_HaningOffRadiator}
\end{subfigure}%
\begin{subfigure}{.5\textwidth}
  \centering
  \includegraphics[width=\linewidth]{Chapter3/Figs/Raster/detCon028b_RadiatorTopDown.png}
  \captionsetup{width=.9\linewidth}
  \caption{}
  \label{subFig:detCon028b_RadiatorTopDown}
\end{subfigure}
\caption{As part of the upgrade, the radiators are significantly larger ensuring the MPPCs are closer to the radiator. In (a) the side A radiator is positioned as the cables are threaded through the insulation on the radiator. In (b) the side A radiator is now in position with all the cables threaded through the insulation.}
\label{fig:detCon_HaningOffRadiator_RadiatorTopDown}
\end{figure}

% \begin{figure}[!h]
% \centering
% \begin{subfigure}{.5\textwidth}
%   \centering
%   \includegraphics[width=\linewidth]{Chapter3/Figs/Raster/detCon029b_coolantFin.png}
%   \captionsetup{width=.9\linewidth}
%   \caption{Board holders that connect the analogue boards to the cooling fins.}
%   \label{subFig:detCon029b_coolantFin}
% \end{subfigure}%
% \begin{subfigure}{.5\textwidth}
%   \centering
%   \includegraphics[width=\linewidth]{Chapter3/Figs/Raster/detCon030b_RadiatorWithFins.png}
%   \captionsetup{width=.9\linewidth}
%   \caption{Cooling fins and board holders attached to the radiator.}
%   \label{subFig:detCon030b_RadiatorWithFins}
% \end{subfigure}
% \caption{The analogue board holders and cooling fins are attached after the cables are threaded through the radiator insulation.}
% \label{fig:detCon_coolantFin_RadiatorWithFins}
% \end{figure}

\begin{figure}[!h]
\centering
\begin{minipage}{.45\textwidth}
  \centering
  \includegraphics[width=\linewidth]{Chapter3/Figs/Raster/detCon030b_RadiatorWithFins.png}
  \captionof{figure}{The radiator on side A with the cooling fins and analogue board holders attached with cables pushed through the insulation and bunched.} 
  \label{fig:detCon030b_RadiatorWithFins}
\end{minipage}%
\qquad
\begin{minipage}{.45\textwidth}
  \centering
  \includegraphics[width=\linewidth]{Chapter3/Figs/Raster/detCon032_ConnectedBoard.png} 
  \captionof{figure}{An analogue board with some of the cables connected to it. A 3D printed cable ``comb'' is used to align the cables.}
  \label{fig:detCon032_ConnectedBoard}
  \vspace{0.478cm} %1 line = 0.478cm % 2 lines = 0.956cm % 3 lines= 1.434cm % 4 lines = 1.912cm % 5 lines = 2.39cm
\end{minipage}
\end{figure}

The upgraded container was delivered September 2019 which can be seen in figure  \ref{subFig:detCon037c_ContainerArrives}. The detector could not be moved into the container until all services especially air conditioning were installed and tested. The upgrade's progress ceased around the same time as the container arrived during September of 2019 as an issue with electronics was discovered. This issue has now been addressed and installation will be able to continue in the near future. This issue has been addressed and although access to the department and services are still somewhat limited by COVID-19 restrictions. The upgraded detector is expected to be completed in 2022. As of November 2020 the detector resides in the upgraded detector (see figure \ref{subFig:detCon039b_PutIn2}).

%Currently, this remains unresolved. Due to the issues in the electronics supply chain caused by the COVID-19 crisis and the small size of the VIDARR collaboration obtaining replacement electronics has been impossible up to September of 2021. Though in recent months there has been some movement in obtaining the relevant components. 

\begin{figure}[!h]
\centering
\begin{subfigure}{.5\textwidth}
  \centering
  \includegraphics[width=\linewidth]{Chapter3/Figs/Raster/detCon037c_ContainerArrives.png}
  \captionsetup{width=.9\linewidth}
  \caption{}
  \label{subFig:detCon037c_ContainerArrives}
\end{subfigure}%
\begin{subfigure}{.5\textwidth}
  \centering
  \includegraphics[width=\linewidth]{Chapter3/Figs/Raster/detCon039b_PutIn2.png}
  \captionsetup{width=.9\linewidth}
  \caption{}
  \label{subFig:detCon039b_PutIn2}
\end{subfigure}
\caption{The new container arrives in (a) and the partially upgraded detector is deposited inside of it in (b).}
\label{fig:detCon_ContainerArrives_PutIn}
\end{figure}

% The detector upgrade has sadly stalled due to the COVID-19. Despite the unprecedented set back the VIDARR collaboration has still been able to increase the detector mass by $\sim$ 50\,\%, assemble and install new radiators, assemble new instruments, cable the MPPCs, and obtain a new container with cooling. Considering this it is reasonable to assume that the collaboration will be able to install the new electrons within good time. 

% The upgraded container has much improved airflow when compared to the original including air conditioning and an improved ventilation system. The upgraded container should be able to keep the temperature consistent and thus reduce the uncertainties caused by temperature fluctuations. In addition, there has been a significant improvement to computational power with an upgraded computer rack to read out detector information and a new computer with 64 total threads and 64\,GB of RAM to assist with analysis. Neither of these are currently in the container due to the issues with COVID-19 previously discussed as they are being used to diagnose any issues with the electronics and analyse older data sets.  

% \begin{figure}[!h]
% \centering
% \includegraphics[width=0.7\linewidth]{Chapter3/Figs/Raster/detCon035b_ContainerAirCon.png}
% \captionof{figure}{The new container has air conditioning and an air circulatory system to help keep a consistent temperature.} 
% \label{fig:detCon035b_ContainerAirCon}
% \end{figure}

% \begin{figure}[!h]
% \centering
% \includegraphics[width=0.8\linewidth]{Chapter3/Figs/Raster/detCon037b_ContainerArrives.png}
% \captionof{figure}{The new container arrives at the University of Liverpool.} 
% \label{fig:detCon037b_ContainerArrives}
% \end{figure}

% \begin{figure}[!h]
% \centering
% \includegraphics[width=0.8\linewidth]{Chapter3/Figs/Raster/detCon039_PutIn2.png}
% \captionof{figure}{The new detector being placed in the new container. (The upgrade was only partially completed at this time)} 
% \label{fig:detCon039_PutIn2}
% \end{figure}

% \begin{figure}[!h]
% \centering
% \includegraphics[width=0.7\linewidth]{Chapter3/Figs/Raster/detCon042b_Rack1.png}
% \captionof{figure}{The computer rack for the upgraded detector from the back (left) and front (right).} 
% \label{fig:detCon042b_Rack1}
% \end{figure}

% \begin{figure}[!h]
% \centering
% \includegraphics[width=0.8\linewidth]{Chapter3/Figs/Raster/detCon044_NewComputer.png}
% \captionof{figure}{The new computer for the upgraded detector. It has two 32 core CPUs and 64\,Gb of Ram.} 
% \label{fig:detCon044_NewComputer}
% \end{figure}

% \begin{figure}[!h]
% \centering
% \includegraphics[width=0.7\linewidth]{Chapter3/Figs/Raster/detCon045b_PowerCabels1.png}
% \captionof{figure}{Power cables for the analogue boards that will attach to the central bus bars. \hl{may need to add some finer details remember to check with Carl}} 
% \label{fig:detCon045b_PowerCabels1}
% \end{figure}